% Created 2021-08-11 Wed 13:04
% Intended LaTeX compiler: lualatex
\documentclass[11pt]{article}
\usepackage{graphicx}
\usepackage{grffile}
\usepackage{longtable}
\usepackage{wrapfig}
\usepackage{rotating}
\usepackage[normalem]{ulem}
\usepackage{amsmath}
\usepackage{textcomp}
\usepackage{amssymb}
\usepackage{capt-of}
\usepackage{hyperref}
\usepackage{tabularx}
\usepackage{etoolbox}
\makeatletter
\def\dontdofcolorbox{\renewcommand\fcolorbox[4][]{##4}}
\AtBeginEnvironment{minted}{\dontdofcolorbox}
\makeatother
\usepackage[newfloat]{minted}
\author{Mark Armstrong}
\date{\today}
\title{Assessment policies}
\hypersetup{
   pdfauthor={Mark Armstrong},
   pdftitle={Assessment policies},
   pdfkeywords={},
   pdfsubject={A living list of notation and conventions used in my principles of programming course.},
   pdfcreator={Emacs 28.0.50 (Org mode 9.4.6)},
   pdflang={English},
   colorlinks,
   linkcolor=blue,
   citecolor=blue,
   urlcolor=blue
   }
\begin{document}

\maketitle
\tableofcontents


\section{Automated unit testing policy}
\label{sec:orga2af092}
Automated unit tests will be provided for all assignments,
and whenever possible and practical,
they will also be provided for each homework.

\textbf{You should not submit the testing files with your assignment contents};
it will not cause any problems, but testing files will be ignored
and overwritten before testing.

Alongside the files for unit testing,
a \href{https://docker.com}{Docker} image
will also be provided, in order to ensure that
you are able to run the tests in the exact same environment
that the course staff will use.

Passing the provided tests is \emph{mandatory},
but \textbf{does not} \emph{guarantee} a passing grade
(both for homeworks and assignments.)
\begin{itemize}
\item Assignments will undergo a code review by the course staff,
and your grade will be influenced just as much or more
by your code's \emph{approach} and \emph{style} as by passing of tests.
\item Homeworks will undergo a similar, but much more cursory, code review.
Barring any obvious issues, you should receive a passing grade
if your code passes the tests.
\end{itemize}
Submissions which do not pass
all or a majority of the tests \textbf{may not be considered for grading at all},
at the discretion of the course staff.

During marking, we will typically add some \emph{additional tests},
often constructed to test what we consider to be
more “extreme” cases than are covered by the the provided tests,
possibly including interesting \href{https://en.wikipedia.org/wiki/Edge\_case}{edge cases}.
\begin{itemize}
\item You are encouraged to try and think of these cases yourselves,
and add appropriate tests to the provided ones
in order to better check your solutions.
\item You are not expected to submit any updates or additions
to the testing files; as mentioned, any submissions of testing files
will be ignored and overwritten.
\end{itemize}

\section{Assignment literate documentation}
\label{sec:orgacc7ec9}
\subsection{Assignment literate documentation policy}
\label{sec:org9ca13e1}
In addition to source code files,
assignments will also require you submit \emph{documentation} for your code
in the form of
a \href{http://www.literateprogramming.com/index.html}{literate programming} document.

20\% of each assignment's marks are set aside for this documentation.
\begin{itemize}
\item 12\% for the contents of the documentation, and
\item 8\% for the style of the documentation.
\item Even if the assignment is incomplete, full documentation marks
may be awarded,
\begin{itemize}
\item so long as some parts are sufficiently completed,
\item and some discussion of the difficulties with missing parts is included.
\begin{itemize}
\item (More than just “I ran out of time”.)
\end{itemize}
\end{itemize}
\end{itemize}

Any of the following formats are acceptable for this documentation:
\begin{itemize}
\item Markdown (\href{https://daringfireball.net/projects/markdown/}{homepage})
\item Org mode (\href{https://orgmode.org/}{homepage})
\begin{itemize}
\item Implementations outside of Emacs exist,
but typically have somewhat minimal features.
\end{itemize}
\item ReStructured text (\href{https://docutils.sourceforge.io/rst.html}{homepage})
\item HTML
\begin{itemize}
\item Using \texttt{<code>} tags,
\item and preferably a tool to provide syntax highlighting,
such as
\begin{itemize}
\item \href{https://highlightjs.org/usage/}{highlight.js} or
\item \href{https://prismjs.com/\#basic-usage}{Prism}
\end{itemize}
\end{itemize}
\item PDF
\begin{itemize}
\item In particular, through \LaTeX{} (\href{https://www.latex-project.org/}{homepage})
\begin{itemize}
\item Using a package such as \texttt{listings}
(\href{https://en.wikibooks.org/wiki/LaTeX/Source\_Code\_Listings}{documentation})
or \texttt{minted}, which provides syntax highlighting
(\href{https://github.com/gpoore/minted}{GitHub homepage and documentation})
\end{itemize}
\end{itemize}
\item Possibly more; speak to us if there is a format you feel
should be accepted.
\begin{itemize}
\item Microsoft Word, OpenOffice and other WSIWYG
(What You See Is What You Get)
editor formats will not be accepted.
\begin{itemize}
\item If you wish to use such an editor, you may export
your file to a PDF for submission.
Do be sure to follow the style guidelines below.
\end{itemize}
\end{itemize}
\end{itemize}

\subsection{Assignment literate documentation style guide}
\label{sec:org7d37673}
As mentioned in the
\hyperref[sec:org9ca13e1]{assignment literate documentation policy},
8\% of the marks of each assignment
is allocated for the style of the documentation.

The section
\hyperref[sec:org08f20a6]{general literate documenation style rules}
below outlines what is required in your documentation,
and what is optional.

Then the section
\hyperref[sec:orgbbd1578]{format specific literate documentation style rules}
 are some comments about style requirements or
recommendations for specific formats.

\subsubsection{General literate documentation style rules}
\label{sec:org08f20a6}
Required:
\begin{itemize}
\item In non-plaintext formats (such as HTML and PDF),
code blocks \textbf{must} be displayed using fixed width
(monospace) fonts.
(\href{https://en.wikipedia.org/wiki/Monospaced\_font}{What's a fixed width font?})
\begin{itemize}
\item In \emph{non}-plaintext formats (such as HTML and PDF),
non-code-block portions should be displayed using
non-fixed width fonts.
\item There \emph{must} be a font distinction between
\end{itemize}
\item Headings (and often subheadings) must be used for organisation.
\begin{itemize}
\item Typically, it is sufficient to use the same document structure
as in the assignment description.
\end{itemize}
\item Code blocks must not be too long; there should be documentation
interspersed with code regularly.
\begin{itemize}
\item There is not a hard and fast rule here;
instead follow these guidelines:
\begin{itemize}
\item Where they are more than a few lines, each function, procedure,
type declaration, etc., should be in its own code block.
\item In no instance should a code block span an entire page or more.
\end{itemize}
\end{itemize}
\end{itemize}

\subsubsection{Format specific literate documentation style rules}
\label{sec:orgbbd1578}
:TODO:

\subsection{Assignment literate documentation content guide}
\label{sec:org53fa978}
:TODO:
\end{document}
