% Created 2020-10-12 Mon 15:49
% Intended LaTeX compiler: lualatex
\documentclass[11pt]{article}
\usepackage{graphicx}
\usepackage{grffile}
\usepackage{longtable}
\usepackage{wrapfig}
\usepackage{rotating}
\usepackage[normalem]{ulem}
\usepackage{amsmath}
\usepackage{textcomp}
\usepackage{amssymb}
\usepackage{capt-of}
\usepackage{hyperref}
\usepackage{tabularx}
\usepackage{etoolbox}
\makeatletter
\def\dontdofcolorbox{\renewcommand\fcolorbox[4][]{##4}}
\AtBeginEnvironment{minted}{\dontdofcolorbox}
\makeatother
\usepackage[newfloat]{minted}
\author{Mark Armstrong}
\date{October 12th, 2020}
\title{Computer Science 3MI3 – 2020 homework 5\\\medskip
\large “Fizzbuzz”-ing in Ruby}
\hypersetup{
   pdfauthor={Mark Armstrong},
   pdftitle={Computer Science 3MI3 – 2020 homework 5},
   pdfkeywords={},
   pdfsubject={Iterating on a simple programming task in Ruby.},
   pdfcreator={Emacs 27.0.90 (Org mode 9.4)},
   pdflang={English},
   colorlinks,
   linkcolor=blue,
   citecolor=blue,
   urlcolor=blue
   }
\begin{document}

\maketitle
\tableofcontents


\section*{Introduction}
\label{sec:org91af0fb}
The “fizzbuzz” problem is a very simple programming task,
sometimes used in interviews to check for a basic understanding
of iterating and branching constructs.

We will investigate various possible approaches to
this problem in Ruby, as a way to become comfortable
with the language.
We begin with the familiar looping statements,
and then move to using “higher-order” methods,
as well as solving a generalisation of the problem.

\section*{Boilerplate}
\label{sec:org3488652}
\subsection*{Submission procedures}
\label{sec:org88ef8ae}
\subsubsection*{Submission method}
\label{sec:org8291ccc}

Homework should be submitted to your McMaster CAS Gitlab respository
in the \texttt{cs3mi3-fall2020} project.

Ensure that you have \textbf{pushed} the commits to the remote repository
in time for the deadline, and not just committed to your local copy.

\subsubsection*{Naming requirements}
\label{sec:orgf9ae364}

Place all files for the homework
inside a folder titled \texttt{hn}, where \texttt{n} is the number of the homework.
So, for homework 1, the use the folder \texttt{h1}, for homework 2 the folder \texttt{h2}, etc.
Ensure you do not capitalise the \texttt{h}.

Unless otherwise instructed in the homework questions,
place all of your code for the homework
in a single file in the \texttt{hn} folder named \texttt{hn.ext},
where \texttt{ext} is the appropriate extension for the language used
according to this list:
\begin{itemize}
\item For Scala, \texttt{ext} is \texttt{sc}.
\item For Prolog, \texttt{ext} is \texttt{pl}.
\item For Ruby, \texttt{ext} is \texttt{rb}.
\item For Clojure, \texttt{ext} is \texttt{clg}.
\end{itemize}
If multiple languages are used in the homework,
submit a \texttt{hn.ext} file for each language.

\begin{center}
\textbf{If the language supports multiple different file extensions,}
\textbf{you must still follow the extension conventions above.}
\end{center}

\begin{center}
\textbf{Incorrect naming of files may result in up to a 10\% deduction in your grade.}
\end{center}

\subsubsection*{Do not submit testing or diagnostic code}
\label{sec:org7f38cda}

Unless you are instructed to do so in the homework questions,
\textbf{you should not submit testing code with your homework submission}.

This includes
\begin{itemize}
\item any \texttt{main} function,
\item any \texttt{print} statements which output information
\textbf{that is not directly requested as console output}
\textbf{in the homework questions}.
\end{itemize}

If you do not wish to remove diagnostic print statements manually,
you will have to find a way to ensure that they disabled
in your final submission.

For instance, by using a wrapper on the print function or macros.

\subsubsection*{Due date and allowance for technical difficulties}
\label{sec:orgbfb008e}

Homework is due on the second Sunday following its release,
by the end of the day (midnight).
Submissions past 00:00 may not be considered.

If you experience technical difficulties
leading up to the submission time,
please contact Mark \textbf{ASAP} with the details of the problem
and, if possible, attach the current state of your homework
to the communication.
This information will help ensure we are able
to accept your submission once the technical difficulties are resolved.

\subsection*{Proper conduct for coursework}
\label{sec:orgab503f1}
\subsubsection*{Individual work}
\label{sec:org7612ef4}

Unless explicitely stated in the homework questions,
all homework in this course is intended
to be \emph{individually completed}.

You are welcome to discuss the content of the homework in
the public forum of the class Microsoft Teams team homework channel,
though obviously solutions or partial solutions should not
be posted or described.

Private discussions about the homework cannot reasonably be
forbidden, but such discussions should follow the same guidelines
as public discussions.

\begin{center}
\textbf{Inappopriate collaboration via private discussions}
\textbf{which is later discovered by course staff}
\textbf{may be considered academic dishonesty.}

When in doubt, make the discussion private, or report its contents
to the course staff by making a note of it
in your homework.
\end{center}

To clarify what is considered appropriate discussions
of homework content, here are some examples:
\begin{enumerate}
\item Discussing the language features introduced or needed for the homework.
\begin{itemize}
\item Such as relevant builtin datatypes
and datatype definition methods and their general use.
\item Code snippets that are not partial solutions to the homework
are welcome and encouraged.
\end{itemize}
\item Questions of the form “What is meant by \texttt{x}?”,
“Does \texttt{x} really mean \texttt{y}?” or “Is there a mistake with \texttt{x}?”
\begin{itemize}
\item Of course, questions of those form which would be answered
by partial solutions are not considered appropriate.
\end{itemize}
\item Questions or advice about errors that may be encountered.
\begin{itemize}
\item Such as “If you see a \texttt{scala.MatchError} you should
probably add a catch-all \texttt{\_} case to your \texttt{match} expressions.”
\end{itemize}
\end{enumerate}

\subsubsection*{Language library resources}
\label{sec:org47b731f}

Unless explicitely stated in the questions,
it is not expected that you will use any language library resources
in the homeworks.

Possible exceptions to this rule include implementations
of datatypes we discuss in this course, such as lists
or options/maybes, if they are included in a standard library
instead of being builtin.

\emph{Basic} operations on such types would also be allowed.
\begin{itemize}
\item For instance, \texttt{head}, \texttt{tail}, \texttt{append}, etc. on lists
would not require explicit permission to be used.
\item More complex operations such as sorting procedures
would require permission before you used them.
\end{itemize}

Additionally, the standard \emph{higher-order} operations
including \texttt{map}, \texttt{reduce}, \texttt{flatten}, and \texttt{filter} are permitted generally,
unless the task is to implement such a higher-order operator.

\section*{Part 1: Fizzbuzzing by loops                            [5 points]}
\label{sec:org5e464be}
In Ruby, create a method \texttt{fizzbuzzLooper} which, given a list
(presumably of integers, though it may contain any type)
creates a new list whose elements
are the elements of the original list
converted into strings, unless they are
\begin{itemize}
\item an integer divisible by both \texttt{3} and \texttt{5}, in which case
they are replaced by \texttt{"fizzbuzz"},
\item an integer divisible by \texttt{3} but not by \texttt{5}, in which case
they are replaced by \texttt{"fizz"}, or
\item an integer divisible by \texttt{5} but not by \texttt{3}, in which case
they are replaced by \texttt{"buzz"}.
\end{itemize}

You may want to make use of the \texttt{to\_s} method on integers;
by convention, \texttt{to\_s} on any type converts
objects of that type into strings.

(Technically, your method should probably work
given any type of collection, not just lists;
but the result should be a list in any case.)

Your \texttt{fizzbuzzLooper} must make use of some manner of
looping construct.
\begin{itemize}
\item Such as a \texttt{loop}, \texttt{while} loop or \texttt{for} loop.
\end{itemize}

\begin{center}
\textbf{Because this is a fairly trivial programming task,}
\textbf{the marking of this question}
\textbf{(and to some extent the marking of the remainder of the homework)
*will take \emph{elegance} more into account than usual.}
(meaning you are expected to follow good coding practices,
especially \emph{not repeating yourself}.)
\end{center}

\section*{Part 2: Fizzbuzzing by iterators (higher-order methods) [10 points]}
\label{sec:orga5ba960}
Construct another method \texttt{fizzbuzzIterator},
whose behaviour is identical to \texttt{fizzbuzzLooper},
but which is defined using an “iterator” method
rather than a looping construct.

See this online
\href{https://ruby-doc.com/docs/ProgrammingRuby/html/tut\_containers.html}{tutorial}
on collections and iterators.
In particular, look into the iterators \texttt{each} and \texttt{collect}.

These iterators take a \emph{block} as argument.
A block is, essentially, Ruby's “lambda” construct.
Blocks are delimited by braces, \texttt{\{\}}, and may have
arguments, which are listed at the beginning and
delimited by pipes, \texttt{||}.
So the anonymous function \texttt{λ x → x + 1} would
be written \texttt{\{ |x| x + 1 \}} in Ruby.

So for instance,
\begin{minted}[breaklines=true]{ruby}
[1,2,3].each { |x| puts(x) }
\end{minted}

outputs each element of the list \texttt{[1,2,3]}.

\section*{Part 3: Generalising fizzbuzzing                        [20 points]}
\label{sec:org4edceea}
We now consider a slight generalisation to
the fizzbuzzing problem, which we will call “zuzzing”.

To generalise the problem, we assume that we may have
several rules which should be applied to the elements of this list,
instead of just the two (if it's disible by 3, output “fizz”,
if it's divisible by 5, output “buzz”.)

We want to create a method which accomodates any number of rules,
and where the rules can be arbitrary predicates
on the elements of the list
(not just “\texttt{λ x → x is divisible by \textasciitilde{}k}”.)

To represent this multitude of rules, we use a list of lists,
assuming each of the lists within the list contain two elements;
\begin{itemize}
\item the first element being a \texttt{lambda} for the rule, and
\item the second element being a \texttt{lambda} for the string
associated with that rule.
\begin{itemize}
\item We use a \texttt{lambda} here as well so that the resulting string
may depend upon the element.
\end{itemize}
\end{itemize}

(The keyword \texttt{lambda} applied to a block allows you to store
that block using a variable, or in our case, in a list;
we are still essentially using blocks in this question.)

For instance, to get the original behaviour of “fizzbuzz”
using this “zuzz”, we would use the rules
\begin{minted}[breaklines=true]{ruby}
rules = [[lambda { |x| x % 3 == 0 }, lambda { |x| "fizz" }],
         [lambda { |x| x % 5 == 0 }, lambda { |x| "buzz" }]]
\end{minted}
as in
\begin{minted}[breaklines=true]{ruby}
zuzzer([1,2,3,4,5,6,7,8,9,10,11,12,13,14,15],rules)
\end{minted}

The reason we use a list of lists of lambdas here to encode the rules,
rather than a hash table or other construct,
is that \emph{the order of the rules matters}.
If more than one rule applies to an element, all such rules
should be applied \emph{in order} to build the resulting string.
For instance, with the “fizzbuzz” rules above, notice
that the “fizz” rule comes before the “buzz” rule
so that if an element is divisible by both 3 and 5,
the result is \texttt{"fizzbuzz"}, not \texttt{"fizz"}, \texttt{"buzz"} or \texttt{"buzzfizz"}.

Create the method \texttt{zuzzer}.

\section*{Part 4: Generalised fizzbuzzing in Scala                [10 bonus points]}
\label{sec:org239204e}
Implement the generalised fizzbuzzing operation from part 3 in Scala.

Make what you feel are necessary adjustments to the types or
implementation details, and describe your choices in comments.
Your solution may be (sometimes subjectively) judged based on
the choices you make. The purpose of the comments is then
for you to convince us your choices are appropriate.

\section*{Part 5: Generalised fizzbuzzing in Prolog               [10 bonus points]}
\label{sec:org4a8fc7b}
Implement the generalised fizzbuzzing operation from part 3 in Scala.

Make what you feel are necessary adjustments to the types or
implementation details, and describe your choices in comments.
Your solution may be (sometimes subjectively) judged based on
the choices you make. The purpose of the comments is then
for you to convince us your choices are appropriate.

\section*{Testing}
\label{sec:org06d8b3b}
:TODO:
\end{document}
