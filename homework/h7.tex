% Created 2020-11-05 Thu 21:54
% Intended LaTeX compiler: lualatex
\documentclass[11pt]{article}
\usepackage{graphicx}
\usepackage{grffile}
\usepackage{longtable}
\usepackage{wrapfig}
\usepackage{rotating}
\usepackage[normalem]{ulem}
\usepackage{amsmath}
\usepackage{textcomp}
\usepackage{amssymb}
\usepackage{capt-of}
\usepackage{hyperref}
\usepackage{tabularx}
\usepackage{etoolbox}
\makeatletter
\def\dontdofcolorbox{\renewcommand\fcolorbox[4][]{##4}}
\AtBeginEnvironment{minted}{\dontdofcolorbox}
\makeatother
\usepackage[newfloat]{minted}
\author{Mark Armstrong}
\date{November 1st, 2020}
\title{Computer Science 3MI3 – 2020 homework 7\\\medskip
\large Pretty printing a representation of the untyped λ-calculus.}
\hypersetup{
   pdfauthor={Mark Armstrong},
   pdftitle={Computer Science 3MI3 – 2020 homework 7},
   pdfkeywords={},
   pdfsubject={Preparing for assignment 2 by writing pretty printers.},
   pdfcreator={Emacs 27.0.90 (Org mode 9.4)},
   pdflang={English},
   colorlinks,
   linkcolor=blue,
   citecolor=blue,
   urlcolor=blue
   }
\begin{document}

\maketitle
\tableofcontents


\section*{Introduction}
\label{sec:org0c8b782}
For this homework, you will extend the implementations
of the \texttt{UL} untyped λ-calculus in Scala and Ruby,
which are provided in assignment 2,
with a \emph{pretty printing} method.

This language implementation makes use of unnamed variables
through \emph{de Bruijn} indices.
This simplifies the implementation; specifically,
\begin{itemize}
\item we eliminate the need to replace names of variables
with “fresh” names during substitution; instead,
\begin{itemize}
\item (a relatively tedious problem;
how do we keep track of fresh names?)
\end{itemize}
\item we must only shift variable indexes according to the
number of enclosing λ's (variable binders.)
\begin{itemize}
\item (a relatively trivial problem.)
\end{itemize}
\end{itemize}

The one major downside to a representation using unnamed variables
is that the terms are far less human readable.
Instead of terms such as
\begin{minted}[breaklines=true]{text}
λ x → λ y → λ z → x y z
\end{minted}
we now have terms such as
\begin{minted}[breaklines=true]{text}
λ λ λ 2 1 0
\end{minted}

Hence, the task of this homework to implement a \emph{pretty printer},
that chooses variable names for a term and “prints” the term
(in fact, it should convert terms to strings,
not print them directly.)

\section*{Updates}
\label{sec:org2bab8d9}
\subsection*{November 5th}
\label{sec:org3b1e9ea}
\begin{itemize}
\item Testing and Docker setup has been added.
\item The suggested output has been adjusted slightly
to account for parentheses being added around
terms being applied to each other.
\item Sample code for how to interact with the \texttt{ULTerm} type
was added to the assignment;
you may want to try it out.
\item The \texttt{ULTerm} code provided for the assignment
has been updated,
\begin{itemize}
\item first to correct a typo with a variable name, and
\item second to add \texttt{toString} methods which result in better
output of the Scala \texttt{ULTerm} type.
\end{itemize}
You may wish to apply those updates to your code as well.
\end{itemize}

\section*{Boilerplate}
\label{sec:org944e768}
\subsection*{Submission procedures}
\label{sec:org6cdf5f4}
\subsubsection*{Submission method}
\label{sec:orga48ff0c}

Homework should be submitted to your McMaster CAS Gitlab respository
in the \texttt{cs3mi3-fall2020} project.

Ensure that you have \textbf{pushed} the commits to the remote repository
in time for the deadline, and not just committed to your local copy.

\subsubsection*{Naming requirements}
\label{sec:orgf1a2a38}

Place all files for the homework
inside a folder titled \texttt{hn}, where \texttt{n} is the number of the homework.
So, for homework 1, use the folder \texttt{h1}, for homework 2 the folder \texttt{h2}, etc.
Ensure you do not capitalise the \texttt{h}.

Unless otherwise instructed in the homework questions,
place all of your code for the homework
in a single file in the \texttt{hn} folder named \texttt{hn.ext},
where \texttt{ext} is the appropriate extension for the language used
according to this list:
\begin{itemize}
\item For Scala, \texttt{ext} is \texttt{sc}.
\item For Prolog, \texttt{ext} is \texttt{pl}.
\item For Ruby, \texttt{ext} is \texttt{rb}.
\item For Clojure, \texttt{ext} is \texttt{clg}.
\end{itemize}
If multiple languages are used in the homework,
submit a \texttt{hn.ext} file for each language.

\begin{center}
\textbf{If the language supports multiple different file extensions,}
\textbf{you must still follow the extension conventions above.}
\end{center}

\begin{center}
\textbf{Incorrect naming of files may result in up to a 10\% deduction in your grade.}
\end{center}

\subsubsection*{Do not submit testing or diagnostic code}
\label{sec:org5c52965}

Unless you are instructed to do so in the homework questions,
\textbf{you should not submit testing code with your homework submission}.

This includes
\begin{itemize}
\item any \texttt{main} function,
\item any \texttt{print} statements which output information
\textbf{that is not directly requested as console output}
\textbf{in the homework questions}.
\end{itemize}

If you do not wish to remove diagnostic print statements manually,
you will have to find a way to ensure that they disabled
in your final submission.

For instance, by using a wrapper on the print function or macros.

\subsubsection*{Due date and allowance for technical difficulties}
\label{sec:org7857e53}

Homework is due on the second Sunday following its release,
by the end of the day (midnight).
Submissions past 00:00 may not be considered.

If you experience technical difficulties
leading up to the submission time,
please contact Mark \textbf{ASAP} with the details of the problem
and, if possible, attach the current state of your homework
to the communication.
This information will help ensure we are able
to accept your submission once the technical difficulties are resolved.

\subsection*{Proper conduct for coursework}
\label{sec:org961f1ca}
\subsubsection*{Individual work}
\label{sec:orge4478df}

Unless explicitely stated in the homework questions,
all homework in this course is intended
to be \emph{individually completed}.

You are welcome to discuss the content of the homework in
the public forum of the class Microsoft Teams team homework channel,
though obviously solutions or partial solutions should not
be posted or described.

Private discussions about the homework cannot reasonably be
forbidden, but such discussions should follow the same guidelines
as public discussions.

\begin{center}
\textbf{Inappopriate collaboration via private discussions}
\textbf{which is later discovered by course staff}
\textbf{may be considered academic dishonesty.}

When in doubt, make the discussion private, or report its contents
to the course staff by making a note of it
in your homework.
\end{center}

To clarify what is considered appropriate discussions
of homework content, here are some examples:
\begin{enumerate}
\item Discussing the language features introduced or needed for the homework.
\begin{itemize}
\item Such as relevant builtin datatypes
and datatype definition methods and their general use.
\item Code snippets that are not partial solutions to the homework
are welcome and encouraged.
\end{itemize}
\item Questions of the form “What is meant by \texttt{x}?”,
“Does \texttt{x} really mean \texttt{y}?” or “Is there a mistake with \texttt{x}?”
\begin{itemize}
\item Of course, questions of those form which would be answered
by partial solutions are not considered appropriate.
\end{itemize}
\item Questions or advice about errors that may be encountered.
\begin{itemize}
\item Such as “If you see a \texttt{scala.MatchError} you should
probably add a catch-all \texttt{\_} case to your \texttt{match} expressions.”
\end{itemize}
\end{enumerate}

\subsubsection*{Language library resources}
\label{sec:org055dc91}

Unless explicitely stated in the questions,
it is not expected that you will use any language library resources
in the homeworks.

Possible exceptions to this rule include implementations
of datatypes we discuss in this course, such as lists
or options/maybes, if they are included in a standard library
instead of being builtin.

\emph{Basic} operations on such types would also be allowed.
\begin{itemize}
\item For instance, \texttt{head}, \texttt{tail}, \texttt{append}, etc. on lists
would not require explicit permission to be used.
\item More complex operations such as sorting procedures
would require permission before you used them.
\end{itemize}

Additionally, the standard \emph{higher-order} operations
including \texttt{map}, \texttt{reduce}, \texttt{flatten}, and \texttt{filter} are permitted generally,
unless the task is to implement such a higher-order operator.

\section*{Part 1: The “pretty printer” \texttt{prettify} method  [30 marks]}
\label{sec:org46140c4}
Implement a method \texttt{prettify} on the \texttt{UTTerm} type
provided in assignment.
The implementation of the \texttt{UTTerm} type can be found
\begin{itemize}
\item \href{./../assignments/src/a2\_ulterm.sc}{here} in Scala, and
\item \href{./../assignments/src/a2\_ulterm.rb}{here} in Ruby.
\end{itemize}

Your method should take a \texttt{UTTerm} are return a string.
As an example, given the \texttt{UTTerm} representing the
λ-term (with unnamed variables)
\begin{minted}[breaklines=true]{text}
λ λ λ 2 (1 0)
\end{minted}
should produce a string such as
\begin{minted}[breaklines=true]{text}
lambda x . lambda y . lambda z . (x) ((y) (z))
\end{minted}

The exact choices of variable names do not matter;
however, you must not change the meaning of the term.
For instance, considering the above term, output
\begin{minted}[breaklines=true]{text}
lambda x . lambda y . lambda x . (x) ((y) (x))
\end{minted}
would be \emph{incorrect}.

(You may want to use a helper method which has an argument to
keep track of the variable names already in use.
But as usual, the implementation details are within your control.)

\section*{Testing}
\label{sec:orga61342e}
Unit tests for the requested methods/functions
are available
\begin{itemize}
\item here \href{./testing/h7/h7t.sc}{h7t.sc} and
\item here \href{./testing/h7/h7t.rb}{h7t.rb}.
\end{itemize}
The contents of the unit test file are also repeated below.

\begin{center}
\textbf{Unlike previous homeworks, passing these tests does not guarantee}
\textbf{your solution behaves as expected.}

The tests will print an example string and the string your method returns.
They do not have to match exactly; instead, the terms they show
must be equivalent except for naming of bound variables.
(This is to allow for variations in how variable names are chosen.)
\end{center}

\subsection*{Automated testing via Docker}
\label{sec:orged92da4}
The Docker setup and usage scripts are available at the following links.
Their contents are also repeated below.
\begin{itemize}
\item \href{./testing/h7/Dockerfile}{Dockerfile}
\item \href{./testing/h7/docker-compose.yml}{docker-compose.yml}
\item \href{./testing/h7/setup.sh}{setup.sh}
\item \href{./testing/h7/run.sh}{run.sh}
\end{itemize}
Place them into your \texttt{h7} directory where your \texttt{h7.sc} and \texttt{h7.rb} files
and the \texttt{h7t.sc} and \texttt{h7t.rb} (linked to above) files exist,
then run \texttt{setup.sh} and \texttt{run.sh}.

You may also refer to the README
for this testing setup and those files
\href{https://github.com/armkeh/principles-of-programming-languages/tree/master/homework/testing/h5}{on the course GitHub repo}.

Note that the use of the \texttt{setup.sh} and \texttt{run.sh} scripts assumes
that you are in a \texttt{bash} like shell; if you are on Windows,
and not using WSL or WSL2, you may have
to run the commands contained in those scripts manually.

\subsection*{The tests}
\label{sec:org58cafb7}
The contents of the testing script are repeated here.

\href{./testing/h7/h7t.sc}{h7t.sc}
\begin{minted}[breaklines=true]{scala}
import $file.h7, h7._

val x = ULVar(0)
val y = ULVar(1)
val z = ULVar(2)
val u = ULVar(3)

val appvars = ULApp(ULApp(x,y), ULApp(z,u))
val lappvars = ULAbs(appvars)
val lllappvars = ULAbs(ULAbs(lappvars))

println("a                                                       Sample")
println(prettify(x))
println("b                                                       Sample")
println(prettify(y))
println("((a) (b)) ((c) (d))                                     Sample")
println(prettify(appvars))
println("lambda a . ((a) (b)) ((c) (d))                          Sample")
println(prettify(lappvars))
println("lambda a . lambda b . lambda c . ((a) (b)) ((c) (d))    Sample")
println(prettify(lllappvars))
\end{minted}

\href{./testing/h7/h7t.rb}{h7t.rb}
\begin{minted}[breaklines=true]{ruby}
require_relative "h7"

x = ULVar.new(0)
y = ULVar.new(1)
z = ULVar.new(2)
u = ULVar.new(3)

appvars = ULApp.new(ULApp.new(x,y), ULApp.new(z,u))
lappvars = ULAbs.new(appvars)
lllappvars = ULAbs.new(ULAbs.new(lappvars))

puts "a                                                       Sample"
puts x.prettify
puts "b                                                       Sample"
puts y.prettify
puts "((a) (b)) ((c) (d))                                     Sample"
puts appvars.prettify
puts "lambda a . ((a) (b)) ((c) (d))                          Sample"
puts lappvars.prettify
puts "lambda a . lambda b . lambda c . ((a) (b)) ((c) (d))    Sample"
puts lllappvars.prettify
\end{minted}

\subsection*{The Docker setup}
\label{sec:orgcc7f258}
The contents of the Docker setup files and scripts are repeated here.

\href{./testing/h7/Dockerfile}{Dockerfile}
\begin{minted}[breaklines=true]{docker}
# Define the argument for openjdk version
ARG OPENJDK_TAG=8u232

FROM ruby:2.7.2-buster

# Setup to install Scala
RUN apt-get update && \
    apt-get install scala -y && \
    apt-get install -y curl && \
    sh -c '(echo "#!/usr/bin/env sh" && \
    curl -L https://github.com/lihaoyi/Ammonite/releases/download/2.1.1/2.12-2.1.1) > /usr/local/bin/amm && \
    chmod +x /usr/local/bin/amm'
RUN (rm -rf /root/.cache)
     
# Set the name of the maintainers
MAINTAINER Habib Ghaffari Hadigheh, Mark Armstrong <ghaffh1@mcmaster.ca, armstmp@mcmaster.ca>

# Set the working directory
WORKDIR /opt/h7
\end{minted}

\href{./testing/h7/docker-compose.yml}{docker-compose.yml}
\begin{minted}[breaklines=true]{yaml}
version: '2'
services:
  service:
    build: .
    image: 3mi3_h7_docker_image
    volumes:
      - .:/opt/h7
    container_name: 3mi3_h7_container
    command: bash -c
      "echo 'Scala testing' &&
       echo '----------------------------------------------------------------------' &&
       amm h7t.sc &&
       printf '\\n\\n\\n' &&
       echo 'Ruby testing' &&
       echo '----------------------------------------------------------------------' &&
       ruby h7t.rb &&
       echo '----------------------------------------------------------------------'"
\end{minted}

\href{./testing/h5/setup.sh}{setup.sh}
\begin{minted}[breaklines=true]{shell}
docker-compose build --force-rm
\end{minted}

\href{./testing/h5/run.sh}{run.sh}
\begin{minted}[breaklines=true]{shell}
# Run the container
docker-compose up --force-recreate
# Stop the container after finishing the test run
docker-compose stop -t 1
\end{minted}

\href{./testing/h5/README.md}{README.md}
\begin{minted}[breaklines=true]{text}
# Instructions for automated testing using Docker

We have already created a `Dockerfile` here which specifies
all the necessary packages, etc., for compiling and running your code.
You only need to follow the instructions below to see 
the results of unit tests designed to check your implementation.

## Setup
We use `docker-compose` and its configuration file to build the image.
Assuming you have `docker` and `docker-compose` installed,
simply execute
```shell script
./setup.sh
```
to generate the image.

## Prepare your code for the running the tests
You only need to place the `h7t.sc` and `h7t.rb` unit test files and
the `run.sh` file in the same directory as your `h7.sc` and `h7.rb` source files.

## Running the tests
As with the build process, we have already put
the configuration needed for running the test inside `docker-compose.yml`.
Simply execute
```shell script
./run.sh
```
to run and see the results of the tests.
\end{minted}
\end{document}
