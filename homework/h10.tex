% Created 2020-12-09 Wed 22:22
% Intended LaTeX compiler: lualatex
\documentclass[11pt]{article}
\usepackage{graphicx}
\usepackage{grffile}
\usepackage{longtable}
\usepackage{wrapfig}
\usepackage{rotating}
\usepackage[normalem]{ulem}
\usepackage{amsmath}
\usepackage{textcomp}
\usepackage{amssymb}
\usepackage{capt-of}
\usepackage{hyperref}
\usepackage{tabularx}
\usepackage{etoolbox}
\makeatletter
\def\dontdofcolorbox{\renewcommand\fcolorbox[4][]{##4}}
\AtBeginEnvironment{minted}{\dontdofcolorbox}
\makeatother
\usepackage[newfloat]{minted}
\author{Mark Armstrong}
\date{December 4, 2020}
\title{Computer Science 3MI3 – 2020 homework 10\\\medskip
\large Using concurrency to break up large problems}
\hypersetup{
   pdfauthor={Mark Armstrong},
   pdftitle={Computer Science 3MI3 – 2020 homework 10},
   pdfkeywords={},
   pdfsubject={Making use of concurrency approaches in all of our languages.},
   pdfcreator={Emacs 27.0.91 (Org mode 9.4)},
   pdflang={English},
   colorlinks,
   linkcolor=blue,
   citecolor=blue,
   urlcolor=blue
   }
\begin{document}

\maketitle
\tableofcontents


\section*{Introduction}
\label{sec:org076be3b}
This homework presents a somewhat contrived problem
involving a computation over a slightly large dataset
in order to justify the use of concurrency.

This problem is based on the one presented on
\href{https://adventofcode.com/2020/day/1}{day 1} of
the \href{https://adventofcode.com/}{Advent of Code} programming challenge
for 2020.

The homework provides a solution to the problem in Clojure, Ruby
and Scala, and asks you to modify the solutions
to take advantage of concurency.

\section*{Boilerplate}
\label{sec:orgffb0d1a}
\subsection*{Submission procedures}
\label{sec:org7861799}
\subsubsection*{Submission method}
\label{sec:orgfe701bb}

Homework should be submitted to your McMaster CAS Gitlab respository
in the \texttt{cs3mi3-fall2020} project.

Ensure that you have \textbf{pushed} the commits to the remote repository
in time for the deadline, and not just committed to your local copy.

\subsubsection*{Naming requirements}
\label{sec:orgce2bc5e}

Place all files for the homework
inside a folder titled \texttt{hn}, where \texttt{n} is the number of the homework.
So, for homework 1, use the folder \texttt{h1}, for homework 2 the folder \texttt{h2}, etc.
Ensure you do not capitalise the \texttt{h}.

Unless otherwise instructed in the homework questions,
place all of your code for the homework
in a single file in the \texttt{hn} folder named \texttt{hn.ext},
where \texttt{ext} is the appropriate extension for the language used
according to this list:
\begin{itemize}
\item For Scala, \texttt{ext} is \texttt{sc}.
\item For Prolog, \texttt{ext} is \texttt{pl}.
\item For Ruby, \texttt{ext} is \texttt{rb}.
\item For Clojure, \texttt{ext} is \texttt{clj}.
\end{itemize}
If multiple languages are used in the homework,
submit a \texttt{hn.ext} file for each language.

\begin{center}
\textbf{If the language supports multiple different file extensions,}
\textbf{you must still follow the extension conventions above.}
\end{center}

\begin{center}
\textbf{Incorrect naming of files may result in up to a 10\% deduction in your grade.}
\end{center}

\subsubsection*{Do not submit testing or diagnostic code}
\label{sec:orgb7c8cea}

Unless you are instructed to do so in the homework questions,
\textbf{you should not submit testing code with your homework submission}.

This includes
\begin{itemize}
\item any \texttt{main} function,
\item any \texttt{print} statements which output information
\textbf{that is not directly requested as console output}
\textbf{in the homework questions}.
\end{itemize}

If you do not wish to remove diagnostic print statements manually,
you will have to find a way to ensure that they disabled
in your final submission.

For instance, by using a wrapper on the print function or macros.

\subsubsection*{Due date and allowance for technical difficulties}
\label{sec:org560021f}

Homework is due on the second Sunday following its release,
by the end of the day (midnight).
Submissions past 00:00 may not be considered.

If you experience technical difficulties
leading up to the submission time,
please contact Mark \textbf{ASAP} with the details of the problem
and, if possible, attach the current state of your homework
to the communication.
This information will help ensure we are able
to accept your submission once the technical difficulties are resolved.

\subsection*{Proper conduct for coursework}
\label{sec:orgde5b39d}
\subsubsection*{Individual work}
\label{sec:orgc063bbc}

Unless explicitely stated in the homework questions,
all homework in this course is intended
to be \emph{individually completed}.

You are welcome to discuss the content of the homework in
the public forum of the class Microsoft Teams team homework channel,
though obviously solutions or partial solutions should not
be posted or described.

Private discussions about the homework cannot reasonably be
forbidden, but such discussions should follow the same guidelines
as public discussions.

\begin{center}
\textbf{Inappopriate collaboration via private discussions}
\textbf{which is later discovered by course staff}
\textbf{may be considered academic dishonesty.}

When in doubt, make the discussion private, or report its contents
to the course staff by making a note of it
in your homework.
\end{center}

To clarify what is considered appropriate discussions
of homework content, here are some examples:
\begin{enumerate}
\item Discussing the language features introduced or needed for the homework.
\begin{itemize}
\item Such as relevant builtin datatypes
and datatype definition methods and their general use.
\item Code snippets that are not partial solutions to the homework
are welcome and encouraged.
\end{itemize}
\item Questions of the form “What is meant by \texttt{x}?”,
“Does \texttt{x} really mean \texttt{y}?” or “Is there a mistake with \texttt{x}?”
\begin{itemize}
\item Of course, questions of those form which would be answered
by partial solutions are not considered appropriate.
\end{itemize}
\item Questions or advice about errors that may be encountered.
\begin{itemize}
\item Such as “If you see a \texttt{scala.MatchError} you should
probably add a catch-all \texttt{\_} case to your \texttt{match} expressions.”
\end{itemize}
\end{enumerate}

\subsubsection*{Language library resources}
\label{sec:orgf359616}

Unless explicitely stated in the questions,
it is not expected that you will use any language library resources
in the homeworks.

Possible exceptions to this rule include implementations
of datatypes we discuss in this course, such as lists
or options/maybes, if they are included in a standard library
instead of being builtin.

\emph{Basic} operations on such types would also be allowed.
\begin{itemize}
\item For instance, \texttt{head}, \texttt{tail}, \texttt{append}, etc. on lists
would not require explicit permission to be used.
\item More complex operations such as sorting procedures
would require permission before you used them.
\end{itemize}

Additionally, the standard \emph{higher-order} operations
including \texttt{map}, \texttt{reduce}, \texttt{flatten}, and \texttt{filter} are permitted generally,
unless the task is to implement such a higher-order operator.

\section*{Part 0: The problem}
\label{sec:org93bf901}
You are provided with a collection of integers for this homework.
What we want to compute on this collection is:
Given an integer \texttt{sum}, find all possible pairs
of (two of) those integers whose sum is less than or equal to \texttt{sum},
and return a sequence of those pairs.

(The pairing of one of the integers with itself is
assumed not to count, unless that integer
appears twice in the provided collection.)

Code is provided which solves this problem in Clojure,
Ruby and Scala. Your task is then to improve the performance
of this code using concurrency.

Note that we will not test for the exact ordering of the sequence,
and it is not required to be in a specific order.
We will instead test for membership within the list.

The collection is available \href{./src/h10/collection.txt}{here} in plaintext,
and below as a defined sequence in each language.

\section*{Part 1: Clojure [15 points]}
\label{sec:org6a15cf1}
The given collection is available in the file \href{./src/h10/collection.clj}{collection.clj},
which defines it as a Clojure array \texttt{input}.

Below, you will see the function \texttt{summingPairs} implemented in Clojure
to solve our given problem.

Implement a version of \texttt{summingPairs} which takes advantage of
the \texttt{future} form to run parts of the computation in separate threads.
You may choose how many threads to make use of.
Try to improve the performance of \texttt{summingPairs} through this process.

Here is the function:
\begin{minted}[breaklines=true]{clojure}
(defn summingPairs [xs sum]
  (letfn [(summingPairsHelper [xs the_pairs]
            ;; If `xs` is empty, we're done.
            (if (empty? xs) the_pairs
                ;; Otherwise, decompose `xs` into the `fst` element
                ;; and the `rest`.
                (let [[fst & rest] xs]
                  ;; We use the `recur` form to make the recursive call.
                  ;; This ensures tail call optimisation
                  (recur
                   rest
                   ;; Concatenate `the_pairs` we have so far with the sequence
                   ;; of every `[fst snd]` where `snd` is in `rest` with
                   ;; `fst + snd <= sum`. The `doall` outside the `concat`
                   ;; forces it to be calculated immediately; without this,
                   ;; we get a (lazy) buildup of `concat`'s which may
                   ;; cause a stack overflow when looking at the result.
                   (doall 
                    (concat the_pairs
                            (for [snd rest ;; For each `snd` in `rest`...
                                  :when (<= (+ fst snd) sum)]
                              ;;... put `[fst snd]` into this sequence.
                              [fst snd])))))))]
    (summingPairsHelper xs [])))
\end{minted}

It can be run over the \texttt{input} using the following code,
which also shows the starting and ending time of the computation,
so you may judge its performance.
\begin{minted}[breaklines=true]{clojure}
(load-file "./collection.clj")

(println (str
          "Starting at:   "
          (.getSecond (java.time.LocalDateTime/now))
          " seconds, "
          (.getNano (java.time.LocalDateTime/now))
          " nanoseconds"))
(println (summingPairs input 2020))
(println (str
          "Ending at:     "
          (.getSecond (java.time.LocalDateTime/now))
          " seconds, "
          (.getNano (java.time.LocalDateTime/now))
          " nanoseconds"))
\end{minted}

\section*{Part 2: Ruby    [15 points]}
\label{sec:orgf5f2f9d}
Repeat part 1 in Ruby, potentially using the below method \texttt{summingPairs} as
a starting point.

The \texttt{input} collection is defined as a Ruby array
in the file \href{./src/h10/collection.rb}{collection.rb}.

The method:
\begin{minted}[breaklines=true]{ruby}
Pair = Struct.new(:fst,:snd)

def summingPairs(xs, sum)
  the_pairs = []
  len = xs.length

  for i in 0..(len-1)
    for j in (i+1)..(len-1)
      if xs[i] + xs[j] <= sum
        the_pairs.push(Pair.new(xs[i],xs[j]))
      end
    end
  end

  return the_pairs
end
\end{minted}

And code to make use of it on the \texttt{input} array:
\begin{minted}[breaklines=true]{ruby}
require 'date'
require_relative 'collection'

puts "Starting at:   #{DateTime.now.sec} seconds, #{DateTime.now.strftime("%9N")} nanoseconds"
puts summingPairs(INPUT,2020)
puts "Ending at:     #{DateTime.now.sec} seconds, #{DateTime.now.strftime("%9N")} nanoseconds"
\end{minted}

\section*{Part 3: Scala   [15 points]}
\label{sec:org0af657c}
Once more, repeat parts 1 and 2, this time in Scala.
You may make use of the below method \texttt{summingPairs} if you like.

This time, the \texttt{input} collection is defined as a Scala array
in \href{./src/h10/collection.sc}{collection.sc}. Note this is an array, not a list as we have
usually made use of in Scala.
The size of the collection necessitated avoiding the (linked) list type.

The method:
\begin{minted}[breaklines=true]{scala}
def summingPairs(xs: Vector[Int], sum: Int): Vector[Tuple2[Int,Int]] = {
  def summingPairsHelper(xs: Vector[Int],
                         the_pairs: Vector[Tuple2[Int,Int]]): Vector[Tuple2[Int,Int]] =
    xs match {
      case fst +: rest =>
        // Search through `rest` for numbers `snd` such that `fst + snd` is the `sum`.
        val pairs_here = rest.collect({case snd if fst + snd <= sum => (fst,snd)})
        // Make the recursive call, adding in the pairs we just found.
        summingPairsHelper(rest, the_pairs ++ pairs_here)
      case _ => the_pairs // If there's no head element, the vector is empty.
    }
  
  summingPairsHelper(xs,Vector())
}
\end{minted}

And code to make use of it on the \texttt{input} array:
\begin{minted}[breaklines=true]{scala}
import java.time.LocalDateTime
import $file.collection, collection._

println(s"Starting at:   ${LocalDateTime.now.getSecond} seconds, ${LocalDateTime.now.getNano} nanoseconds")
println(summingPairs(input,2020))
println(s"Ending at:     ${LocalDateTime.now.getSecond} seconds, ${LocalDateTime.now.getNano} nanoseconds")
\end{minted}

\section*{Part 4: Prolog  [5 bonus points]}
\label{sec:org5e3cddd}
For bonus marks, implement a \texttt{summingPairs} predicate in Prolog,
making use of the \texttt{concurrent} predicate.

\section*{Testing}
\label{sec:org8d4980a}
Unit tests for the requested methods/functions
are available:
\begin{itemize}
\item \href{./testing/h10/h10t.clj}{h10t.clj} for Clojure,
\item \href{./testing/h10/h10t.rb}{h10t.rb} for Ruby, and
\item \href{./testing/h10/h10t.sc}{h10t.sc} for Scala.
\end{itemize}
The contents of the unit test file are also repeated below.

\subsection*{Automated testing via Docker}
\label{sec:org2045f18}
The Docker setup and usage scripts are available at the following links.
Their contents are also repeated below.
\begin{itemize}
\item \href{./testing/h10/Dockerfile}{Dockerfile}
\item \href{./testing/h10/docker-compose.yml}{docker-compose.yml}
\item \href{./testing/h10/setup.sh}{setup.sh}
\item \href{./testing/h10/run.sh}{run.sh}
\end{itemize}
Place them into your \texttt{h10} directory where your \texttt{h10.clj} file
and the \texttt{h10t.clj} (linked to above) files exist,
then run \texttt{setup.sh} and \texttt{run.sh}.

You may also refer to the README
for this testing setup and those files
\href{https://github.com/armkeh/principles-of-programming-languages/tree/master/homework/testing/h10}{on the course GitHub repo}.

Note that the use of the \texttt{setup.sh} and \texttt{run.sh} scripts assumes
that you are in a \texttt{bash} like shell; if you are on Windows,
and not using WSL or WSL2, you may have
to run the commands contained in those scripts manually.

\subsection*{The tests}
\label{sec:org2e2144f}
The contents of the testing scripts are repeated here.

\href{./testing/h10/h10t.clj}{h10t.clj}
\begin{minted}[breaklines=true]{clojure}
(ns testing)
(use 'clojure.test)
(load-file "collection.clj") ;; The `collection` file provided with the homework.
(load-file "h10.clj")

(deftest contains-checks
  (let [result (summingPairs input 2020)]
    (is (contains? (set result) [374 150]))
    (is (contains? (set result) [626 1030]))
    (is (contains? (set result) [150 1850]))))

;; If we define `test-ns-hook`, it is called when running `run-tests`,
;; instead of just calling all tests in the namespace.
;; This lets us control the order of the tests.
(deftest test-ns-hook
  (contains-checks))

(run-tests 'testing)
\end{minted}

\href{./testing/h10/h10t.rb}{h10t.rb}
\begin{minted}[breaklines=true]{ruby}
require_relative "h10"
require_relative "collection"
require "test/unit"

RESULT = summingPairs(INPUT,2020)

class SimpleTests < Test::Unit::TestCase
  def test_contains_checks
    assert_equal(true, RESULT.include?(Pair.new(374, 150)))
    assert_equal(true, RESULT.include?(Pair.new(626, 1030)))
    assert_equal(true, RESULT.include?(Pair.new(150, 1850)))
  end
end
\end{minted}

\href{./testing/h10/h10t.sc}{h10t.sc}
\begin{minted}[breaklines=true]{scala}
import $file.h10, h10._
import $file.collection, collection._

/* Given an expected result and a computed result,
   check if they are equal in value.
   If so, return 0. Otherwise, inform the user, and return 1,
   so the number of failures can be counted. */
def test[A](given: A, expected: A, the_test: String) =
  if (!(given equals expected)) {
    println("+---------------------------------------------------")
    println("| " + the_test + " failed.")
    println("| Expected " + expected + ", got " + given + ".")
    println("+---------------------------------------------------")
    1
  } else {
    0
  }


val result = summingPairs(input,2020)

//  The tests are saved as tuples, the pieces of which will be passed
// to test.
val tests = List(
  (result.contains(Tuple2(374,150)), true, "374,150"),
  (result.contains(Tuple2(626,1030)), true, "626,1030"),
  (result.contains(Tuple2(150,1850)), true, "150,1850"))

// Apply test to each element of tests, and sum the return values.
// This is essentially a for loop.
val failed = tests.foldLeft(0) {
  (failures, next) => next match {
    // Deconstruct the tuple to get its parts
    case (given, expected, the_test) => failures + test(given, expected, the_test)
  }
}

println("+---------------------------------------------------")
println("| " + failed + " tests failed")
println("+---------------------------------------------------")
\end{minted}

\href{./testing/h10/h10ta.sc}{h10ta.sc}; an alternative version of the above tests
which expects a \texttt{Future}-wrapped value from \texttt{summingPairs}.
\begin{minted}[breaklines=true]{scala}
import $file.h10a, h10a._
import $file.collection, collection._
import scala.concurrent.Future
import scala.concurrent.ExecutionContext.Implicits.global
import scala.util.{Try, Success, Failure}
import scala.concurrent._
import scala.concurrent.duration._

/* Given an expected result and a computed result,
   check if they are equal in value.
   If so, return 0. Otherwise, inform the user, and return 1,
   so the number of failures can be counted. */
def test[A](given: A, expected: A, the_test: String) =
  if (!(given equals expected)) {
    println("+---------------------------------------------------")
    println("| " + the_test + " failed.")
    println("| Expected " + expected + ", got " + given + ".")
    println("+---------------------------------------------------")
    1
  } else {
    0
  }


val result = summingPairs(input,2020)

// The rest of the script is mapped over the `Future`.
result.onComplete({ case Success(result) =>
  // The tests are saved as tuples, the pieces of which will be passed
  // to test.
  val tests = List(
    (result.contains(Tuple2(374,150)), true, "374,150"),
    (result.contains(Tuple2(626,1030)), true, "626,1030"),
    (result.contains(Tuple2(150,1850)), true, "150,1850"))
  
  // Apply test to each element of tests, and sum the return values.
  // This is essentially a for loop.
  val failed = tests.foldLeft(0) {
    (failures, next) => next match {
      // Deconstruct the tuple to get its parts
      case (given, expected, the_test) => failures + test(given, expected, the_test)
    }
  }

  println("+---------------------------------------------------")
  println("| " + failed + " tests failed")
  println("+---------------------------------------------------")
})
\end{minted}


\subsection*{The Docker setup}
\label{sec:orga511aa6}
The contents of the Docker setup files and scripts are repeated here.

\href{./testing/h10/Dockerfile}{Dockerfile}
\begin{minted}[breaklines=true]{docker}
# Define the argument for openjdk version
ARG OPENJDK_TAG=8u232

FROM clojure:openjdk-8

# Setup to install Scala
RUN apt-get update && \
    apt-get install scala -y && \
    apt-get install -y curl && \
    sh -c '(echo "#!/usr/bin/env sh" && \
    curl -L https://github.com/lihaoyi/Ammonite/releases/download/2.1.1/2.12-2.1.1) > /usr/local/bin/amm && \
    chmod +x /usr/local/bin/amm'
RUN (rm -rf /root/.cache)

# Install Ruby
RUN apt-get update && apt-get install -y --no-install-recommends --no-install-suggests curl bzip2 build-essential libssl-dev libreadline-dev zlib1g-dev && \
    rm -rf /var/lib/apt/lists/* && \
    curl -L https://github.com/rbenv/ruby-build/archive/v20201118.tar.gz | tar -zxvf - -C /tmp/ && \
    cd /tmp/ruby-build-* && ./install.sh && cd / && \
    ruby-build -v 2.7.2 /usr/local && rm -rfv /tmp/ruby-build-*

# Set the name of the maintainers
MAINTAINER Habib Ghaffari Hadigheh, Mark Armstrong <ghaffh1@mcmaster.ca, armstmp@mcmaster.ca>

# Set the working directory
WORKDIR /opt/h10
\end{minted}

\href{./testing/h10/docker-compose.yml}{docker-compose.yml}
\begin{minted}[breaklines=true]{yaml}
version: '2'
services:
  service:
    build: .
    image: 3mi3_h10_docker_image
    volumes:
      - .:/opt/h10
    container_name: 3mi3_h10_container
    command: bash -c
      "echo 'Scala testing' ;
       echo '----------------------------------------------------------------------' ;
       timeout 2m amm h10t.sc ;
       printf '\\n\\n\\n' ;
       echo 'Scala testing (alternate)' ;
       echo '----------------------------------------------------------------------' ;
       timeout 2m amm h10ta.sc ;
       printf '\\n\\n\\n' ;
       echo 'Ruby testing' ;
       echo '----------------------------------------------------------------------' ;
       timeout 2m ruby h10t.rb ;
       echo '----------------------------------------------------------------------' ;
       printf '\\n\\n\\n' ;
       echo 'Clojure testing' ;
       echo '----------------------------------------------------------------------' ;
       cat h10t.clj | timeout 2m lein repl ;
       echo '----------------------------------------------------------------------'"
\end{minted}

\href{./testing/h10/setup.sh}{setup.sh}
\begin{minted}[breaklines=true]{shell}
docker-compose build --force-rm
\end{minted}

\href{./testing/h10/run.sh}{run.sh}
\begin{minted}[breaklines=true]{shell}
# Run the container
docker-compose up --force-recreate
# Stop the container after finishing the test run
docker-compose stop -t 1
\end{minted}

\href{./testing/h10/README.md}{README.md}
\begin{minted}[breaklines=true]{text}
# Instructions for automated testing using Docker

We have already created a `Dockerfile` here which specifies
all the necessary packages, etc., for compiling and running your code.
You only need to follow the instructions below to see 
the results of unit tests designed to check your implementation.

## Setup
We use `docker-compose` and its configuration file to build the image.
Assuming you have `docker` and `docker-compose` installed,
simply execute
```shell script
./setup.sh
```
to generate the image.

## Prepare your code for the running the tests
You only need to place the `h8t.clj` unit test file and
the `run.sh` file in the same directory as your `h9.clj` source file.

## Running the tests
As with the build process, we have already put
the configuration needed for running the test inside `docker-compose.yml`.
Simply execute
```shell script
./run.sh
```
to run and see the results of the tests.
\end{minted}
\end{document}
