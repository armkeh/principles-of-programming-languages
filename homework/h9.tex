% Created 2020-12-08 Tue 01:43
% Intended LaTeX compiler: lualatex
\documentclass[11pt]{article}
\usepackage{graphicx}
\usepackage{grffile}
\usepackage{longtable}
\usepackage{wrapfig}
\usepackage{rotating}
\usepackage[normalem]{ulem}
\usepackage{amsmath}
\usepackage{textcomp}
\usepackage{amssymb}
\usepackage{capt-of}
\usepackage{hyperref}
\usepackage{tabularx}
\usepackage{etoolbox}
\makeatletter
\def\dontdofcolorbox{\renewcommand\fcolorbox[4][]{##4}}
\AtBeginEnvironment{minted}{\dontdofcolorbox}
\makeatother
\usepackage[newfloat]{minted}
\author{Mark Armstrong}
\date{November 28th, 2020}
\title{Computer Science 3MI3 – 2020 homework 9\\\medskip
\large Adding “guarded commands” to Clojure}
\hypersetup{
   pdfauthor={Mark Armstrong},
   pdftitle={Computer Science 3MI3 – 2020 homework 9},
   pdfkeywords={},
   pdfsubject={Exploring a different kind of control structure: the guarded command.},
   pdfcreator={Emacs 27.0.91 (Org mode 9.4)},
   pdflang={English},
   colorlinks,
   linkcolor=blue,
   citecolor=blue,
   urlcolor=blue
   }
\begin{document}

\maketitle
\tableofcontents


\section*{Introduction}
\label{sec:orgb06f954}
In his 1975 paper
\href{https://dl.acm.org/doi/10.1145/360933.360975}{Guarded commands, nondeterminacy and formal derivation of programs},
Edsger W. Dijkstra proposed a foundation for an imperative language
different from the commonly used branching and iterating constructs;
the \emph{guarded commands}, along with control structures operating on them.

The guarded command language is especially interesting
in comparison to the languages we have developed
in that it is inherently \emph{non-deterministic}.

In this homework, we familiarise ourselves with the guarded command
control constructs by implementing them in Clojure, using macros.

\section*{Boilerplate}
\label{sec:orgb3b2153}
\subsection*{Submission procedures}
\label{sec:org9034036}
\subsubsection*{Submission method}
\label{sec:orgf9cf8c8}

Homework should be submitted to your McMaster CAS Gitlab respository
in the \texttt{cs3mi3-fall2020} project.

Ensure that you have \textbf{pushed} the commits to the remote repository
in time for the deadline, and not just committed to your local copy.

\subsubsection*{Naming requirements}
\label{sec:org39d1bae}

Place all files for the homework
inside a folder titled \texttt{hn}, where \texttt{n} is the number of the homework.
So, for homework 1, use the folder \texttt{h1}, for homework 2 the folder \texttt{h2}, etc.
Ensure you do not capitalise the \texttt{h}.

Unless otherwise instructed in the homework questions,
place all of your code for the homework
in a single file in the \texttt{hn} folder named \texttt{hn.ext},
where \texttt{ext} is the appropriate extension for the language used
according to this list:
\begin{itemize}
\item For Scala, \texttt{ext} is \texttt{sc}.
\item For Prolog, \texttt{ext} is \texttt{pl}.
\item For Ruby, \texttt{ext} is \texttt{rb}.
\item For Clojure, \texttt{ext} is \texttt{clj}.
\end{itemize}
If multiple languages are used in the homework,
submit a \texttt{hn.ext} file for each language.

\begin{center}
\textbf{If the language supports multiple different file extensions,}
\textbf{you must still follow the extension conventions above.}
\end{center}

\begin{center}
\textbf{Incorrect naming of files may result in up to a 10\% deduction in your grade.}
\end{center}

\subsubsection*{Do not submit testing or diagnostic code}
\label{sec:orgbf339fb}

Unless you are instructed to do so in the homework questions,
\textbf{you should not submit testing code with your homework submission}.

This includes
\begin{itemize}
\item any \texttt{main} function,
\item any \texttt{print} statements which output information
\textbf{that is not directly requested as console output}
\textbf{in the homework questions}.
\end{itemize}

If you do not wish to remove diagnostic print statements manually,
you will have to find a way to ensure that they disabled
in your final submission.

For instance, by using a wrapper on the print function or macros.

\subsubsection*{Due date and allowance for technical difficulties}
\label{sec:org6b87735}

Homework is due on the second Sunday following its release,
by the end of the day (midnight).
Submissions past 00:00 may not be considered.

If you experience technical difficulties
leading up to the submission time,
please contact Mark \textbf{ASAP} with the details of the problem
and, if possible, attach the current state of your homework
to the communication.
This information will help ensure we are able
to accept your submission once the technical difficulties are resolved.

\subsection*{Proper conduct for coursework}
\label{sec:orgfe223ba}
\subsubsection*{Individual work}
\label{sec:org6840550}

Unless explicitely stated in the homework questions,
all homework in this course is intended
to be \emph{individually completed}.

You are welcome to discuss the content of the homework in
the public forum of the class Microsoft Teams team homework channel,
though obviously solutions or partial solutions should not
be posted or described.

Private discussions about the homework cannot reasonably be
forbidden, but such discussions should follow the same guidelines
as public discussions.

\begin{center}
\textbf{Inappopriate collaboration via private discussions}
\textbf{which is later discovered by course staff}
\textbf{may be considered academic dishonesty.}

When in doubt, make the discussion private, or report its contents
to the course staff by making a note of it
in your homework.
\end{center}

To clarify what is considered appropriate discussions
of homework content, here are some examples:
\begin{enumerate}
\item Discussing the language features introduced or needed for the homework.
\begin{itemize}
\item Such as relevant builtin datatypes
and datatype definition methods and their general use.
\item Code snippets that are not partial solutions to the homework
are welcome and encouraged.
\end{itemize}
\item Questions of the form “What is meant by \texttt{x}?”,
“Does \texttt{x} really mean \texttt{y}?” or “Is there a mistake with \texttt{x}?”
\begin{itemize}
\item Of course, questions of those form which would be answered
by partial solutions are not considered appropriate.
\end{itemize}
\item Questions or advice about errors that may be encountered.
\begin{itemize}
\item Such as “If you see a \texttt{scala.MatchError} you should
probably add a catch-all \texttt{\_} case to your \texttt{match} expressions.”
\end{itemize}
\end{enumerate}

\subsubsection*{Language library resources}
\label{sec:orgdac2e12}

Unless explicitely stated in the questions,
it is not expected that you will use any language library resources
in the homeworks.

Possible exceptions to this rule include implementations
of datatypes we discuss in this course, such as lists
or options/maybes, if they are included in a standard library
instead of being builtin.

\emph{Basic} operations on such types would also be allowed.
\begin{itemize}
\item For instance, \texttt{head}, \texttt{tail}, \texttt{append}, etc. on lists
would not require explicit permission to be used.
\item More complex operations such as sorting procedures
would require permission before you used them.
\end{itemize}

Additionally, the standard \emph{higher-order} operations
including \texttt{map}, \texttt{reduce}, \texttt{flatten}, and \texttt{filter} are permitted generally,
unless the task is to implement such a higher-order operator.

\section*{Part 0.1: An introduction to guarded commands}
\label{sec:orgf549d6d}
A very brief and informative article discussing the guarded command
by Jerrold L. Wagener is freely available
from the \href{https://dl.acm.org/doi/10.5555/1074100.1074433}{ACM digital library}.

A \emph{guarded command} consists of a (presumably boolean valued) \emph{guard} along
with a \emph{command} (an expression or statement of the language).

We write a set or sequence of guarded commands as
\begin{minted}[breaklines=true]{text}
  B₁ ⟶ S₁
  B₂ ⟶ S₂
     ⋮
  Bₙ ⟶ Sₙ
\end{minted}
(where each \texttt{Bᵢ} is the guard for the command \texttt{Sᵢ}).

By itself, a set of guarded commands is not a control structure.
Instead, we introduce special constructs
which operate on sets of guarded commands to form a control structure.

The \texttt{if} construct, when applied to a set of guarded commands, as in
\begin{minted}[breaklines=true]{text}
if
  B₁ ⟶ S₁
  B₂ ⟶ S₂
     ⋮
  Bₙ ⟶ Sₙ
\end{minted}
selects any command whose guard evaluates to true,
and executes that command.
If no command is true, it does nothing.

The \texttt{do} construct, when applied to a set of guarded commands, as in
\begin{minted}[breaklines=true]{text}
do
  B₁ ⟶ S₁
  B₂ ⟶ S₂
     ⋮
  Bₙ ⟶ Sₙ
\end{minted}
also selects any command whose guard evaluates to true
and executes that command,
but it \emph{continues to do so} until no guard is true.
That is, the \texttt{do} is a loop, which at each step executes a random command
(with a valid guard) and only stops when all of the guards are \texttt{false}.

We will be interested in versions of the \texttt{if} and \texttt{do} constructs
which \emph{nondeterministically} select which command to evaluate.
This can be accomplished by making use of functions
which act \emph{randomly}.

The appeal of this nondeterminism is that it forces the programmer
to be certain that their programs behaviour does not depend
upon the ordering of the commands;
instead, the guards must be made explicit enough
to ensure that their command is only executed in the correct context.

\section*{Part 0.2: Representing guarded commands in Clojure}
\label{sec:orge6793cb}
We can easily write guarded commands in Clojure
as a record consisting of the \texttt{guard} and the \texttt{command}.
\begin{minted}[breaklines=true]{clojure}
(defrecord GuardedCommand [guard command])
\end{minted}

We can create an instance of this record as in
\begin{minted}[breaklines=true]{clojure}
(GuardedCommand. '(> x 5) '(- x 1))
\end{minted}

We can access the fields of these records as if they were maps,
or using syntax based on Java field accessors.
That is, given a \texttt{GuardedCommand} instance \texttt{grd-cmd},
we can write e.g., \texttt{(:guard grd-cmd)} or \texttt{(.command grd-cmd)}.
\begin{minted}[breaklines=true]{clojure}
(let [grd-cmd (GuardedCommand. '(> x 5) '(- x 1))]
  (printf "The guard of this command is %s\n" (:guard grd-cmd))
  (printf "The command of this command is %s\n" (.command grd-cmd)))
\end{minted}

So we will operate on lists or vectors of \texttt{GuardCommands}.

\section*{Part 0.3: An example construct for using guarded commands}
\label{sec:org61256ce}
To get you started, we define here a \emph{deterministic} \texttt{if} construct
operating on guarded commands.

This construct differs from the one you will be tasked to define
in that it always chooses the first command in the sequence
whose guard is true
(instead of nondeterministically/randomly selecting a command.)

\begin{minted}[breaklines=true]{clojure}
(defn first-allowed-command
  "Find the first command in a sequence of guarded `commands`
whose `.guard` evaluates to a truthy value and return its `.command`.
Returns `nil` if none of the guards are satisfied."
  [commands]
  ;; If the `commands` list is empty, "do nothing" by returning `nil`.
  (if (empty? commands) nil
      ;; Otherwise, deconstruct the `commands` list into
      ;; the first `command` and the `rest`.
      (let [[command & rest] commands]
        ;; Diagnostic print statement, if needed.
        ;(printf "Checking command %s with guard %s and command %s\n" command (.guard command) (.command command))
        ;; Now check the `guard`, and if it's satisfied, return the first `command`.
        (if (eval (.guard command)) (.command command)
            ;; Otherwise, continue to check the `rest` of the guarded commands.
            (first-allowed-command rest)))))

(defmacro guarded-deterministic-if
  "Given a sequence of `GuardedCommands`, `commands`,
select the first guarded command whose `.guard` evaluates
to a truthy value and evaluate its `.command`."
  [& commands]
  ;; The body must be quoted, so that nothing is evaluated until runtime.
  `(eval ;; Evaluate...
    (first-allowed-command ;; ...the command returned by first-allowed-command... 
     [~@commands]))) ;; to which we pass a vector of the commands.
;; The ~@ applied to `commands` here "splices" the elements of `commands` into place here.
;; That is, each element of the sequence `commands` is inserted here in order.
;; But not literally as a sequence (between parentheses or brackets.) Hence we wrap in [].
;; The use of the [] is actually quite particular; using a quoted list, '(...), would not work.
;; Because the guarded commands within would be treated as sequences instead of records.
\end{minted}

As an example, we use this form to define a \texttt{max} operation.
\begin{minted}[breaklines=true]{clojure}
(defn max [x y]
  (guarded-deterministic-if
   ;; For variables to maintain their meaning within a quoted list,
   ;; use the special backtick ` quote and unquote the variables with ~.
   (GuardedCommand. `(>= ~x ~y) x)
   (GuardedCommand. `(>= ~y ~x) y)))
\end{minted}

\section*{Part 1: Sequence of commands whose guards are satisfied       [20 points]}
\label{sec:org45a647e}
Create a function \texttt{allowed-commands} which, given a sequence
of \texttt{GuardedCommands} (the record type defined in Part 0.2)
produces a list of commands whose guard is satisfied
(evaluates to a truthy value.)

For instance,
\begin{minted}[breaklines=true]{clojure}
(let [x 10
      y 10]
  (allowed-commands
   [(GuardedCommand. `(>= ~x ~y) `(printf "%s is greater than or equal to %s" ~x ~y))
    (GuardedCommand. `(=  ~x ~y) `(printf "%s is equal to %s" ~x ~y))
    (GuardedCommand. `(<= ~x ~y) `(printf "%s is less than or equal to %s" ~x ~y))]))
\end{minted}
should return a sequence of all three of the commands
(since \texttt{x} is equal to \texttt{y} here.)
Whereas
\begin{minted}[breaklines=true]{clojure}
(let [x  5
      y 10]
  (allowed-commands
   [(GuardedCommand. `(>= ~x ~y) `(printf "%s is greater than or equal to %s" ~x ~y))
    (GuardedCommand. `(=  ~x ~y) `(printf "%s is equal to %s" ~x ~y))
    (GuardedCommand. `(<= ~x ~y) `(printf "%s is less than or equal to %s" ~x ~y))]))
\end{minted}
should return a sequence containing only the last command
(since \texttt{x} is strictly less than \texttt{y} here.)

(Refer to the \texttt{first-allowed-command} function defined in part 0.3
as a possible starting point for your \texttt{allowed-commands} function.
Other approaches are permitted and encouraged, though.)

\section*{Part 2: A nondeterministic \texttt{if} expression for guarded commands [15 points]}
\label{sec:orgd34fcda}
Define a macro for the nondeterministic \texttt{if} construct
called \texttt{guarded-if}.

It should take a sequence of \texttt{GuardedCommand} instances,
randomly pick one whose guard is true,
and execute its command.

The \texttt{rand-nth} function documented \href{https://clojuredocs.org/clojure.core/rand-nth}{here},
which picks a random element out of a sequence, may be of use here
(in conjunction with your function from part 1.)

\section*{Part 3: A nondeterministic \texttt{do} expression for guarded commands [15 points]}
\label{sec:org06b3ee2}
Now define a macro for the \textbf{looping} nondeterministic \texttt{do} construct
called \texttt{guarded-do}. In contrast to the \texttt{guarded-if} macro,
this construct should continue to loop and evaluate commands
until none of the guards are true.

(Added December 7th.) We have not discussed how to create and
handle mutable (non-constant) variables in Clojure.
And in fact, because of the way this assignment instructs
you to represent guarded commands (as a record),
it actually might be impossible to use your macro on Clojure code
that involves mutable variables.
Because of this, you may only be able to test out your \texttt{guarded-do} macro
by causing an infinite loop.

Because of this, we will not actually be testing the \texttt{guarded-do};
it will be checked by visual inspection only.

\section*{Part 4: GCD                                                   [10 points]}
\label{sec:org32b3680}
Use the guarded command constructs you have defined
to define a function \texttt{gcd} to find the greatest common denominator
of two integers.

The intention for this part is that you use the \texttt{if} construct
from part 2 (corrected December 8th; this originally said part 1)
and recursion to define the GCD function.
A version using iteration (the \texttt{do} construct) is given as a bonus.

Note that the iterative algorithm for the GCD using guarded commands is
very well known; it was the first example used by Dijkstra
in his presentation of the language.
You can see this algorithm in
\href{https://dl.acm.org/doi/10.5555/1074100.1074433}{Wagener's paper} referenced above.
It should be relatively simple to translate this to a recursive algorithm.

\section*{Part 5: GCD by iteration [5 bonus points]}
\label{sec:orgb47799f}
Define the function \texttt{gcd-iter} which calculates the GCD of two integers
using iteration (the \texttt{do} construct.)

The challenge to this part is not the algorithm;
instead, it is the use of mutable variables,
which we have not shown in Clojure.

\section*{Testing}
\label{sec:org6df55d6}
Unit tests for the requested methods/functions
are available here \href{./testing/h9/h9t.clj}{h9t.clj}.
The contents of the unit test file are also repeated below.

\subsection*{Automated testing via Docker}
\label{sec:org82bb06d}
The Docker setup and usage scripts are available at the following links.
Their contents are also repeated below.
\begin{itemize}
\item \href{./testing/h9/Dockerfile}{Dockerfile}
\item \href{./testing/h9/docker-compose.yml}{docker-compose.yml}
\item \href{./testing/h9/setup.sh}{setup.sh}
\item \href{./testing/h9/run.sh}{run.sh}
\end{itemize}
Place them into your \texttt{h9} directory where your \texttt{h9.clj} file
and the \texttt{h9t.clj} (linked to above) files exist,
then run \texttt{setup.sh} and \texttt{run.sh}.

You may also refer to the README
for this testing setup and those files
\href{https://github.com/armkeh/principles-of-programming-languages/tree/master/homework/testing/h5}{on the course GitHub repo}.

Note that the use of the \texttt{setup.sh} and \texttt{run.sh} scripts assumes
that you are in a \texttt{bash} like shell; if you are on Windows,
and not using WSL or WSL2, you may have
to run the commands contained in those scripts manually.

\subsection*{The tests}
\label{sec:org3a457b2}
The contents of the testing script are repeated here.

\href{./testing/h9/h9t.clj}{h9t.clj}
\begin{minted}[breaklines=true]{clojure}
(ns testing)
(use 'clojure.test)
(load-file "h9.clj")


;; Tests based on the examples from the homework.
(def compare-10-10
  (let [x 10
        y 10]
    [(GuardedCommand. `(>= ~x ~y) (str x " is >= " y))
     (GuardedCommand. `(=  ~x ~y) (str x " is =  " y))
     (GuardedCommand. `(<= ~x ~y) (str x " is <= " y))]))

(def compare-5-10
  (let [x  5
        y 10]
    [(GuardedCommand. `(>= ~x ~y) (str x " is >= " y))
     (GuardedCommand. `(=  ~x ~y) (str x " is =  " y))
     (GuardedCommand. `(<= ~x ~y) (str x " is <= " y))]))

(def compare-10-5
  (let [x 10
        y  5]
    [(GuardedCommand. `(>= ~x ~y) (str x " is >= " y))
     (GuardedCommand. `(=  ~x ~y) (str x " is =  " y))
     (GuardedCommand. `(<= ~x ~y) (str x " is <= " y))]))



;; A method to compute a given expression a number of times,
;; creating a list of the results.
;; Used to test the non-determinacy of the `guarded-if` and `guarded-do` forms.
(defn times [n f]
  "Repeatedly evaluate `f` `n` times and concatenate together the results."
  (concat
   (repeatedly n #(eval f))))



(deftest allowed-commands-compare-10-10
  (is (some #(= "10 is >= 10" %) (allowed-commands compare-10-10))
            "When comparing 10 and 10, the \">=\" command was not allowed!")
  (is (some #(= "10 is =  10" %) (allowed-commands compare-10-10))
            "When comparing 10 and 10, the \"=\" command was not allowed!")
  (is (some #(= "10 is <= 10" %) (allowed-commands compare-10-10))
            "When comparing 10 and 10, the \"<=\" command was not allowed!"))

(deftest allowed-commands-compare-5-10
  (is (not (some #(= "5 is >= 10" %) (allowed-commands compare-5-10)))
            "When comparing 5 and 10, the \">=\" command was incorrectly allowed!")
  (is (not (some #(= "5 is =  10" %) (allowed-commands compare-5-10)))
            "When comparing 5 and 10, the \"=\" command was incorrectly allowed!")
  (is      (some #(= "5 is <= 10" %) (allowed-commands compare-5-10))
            "When comparing 5 and 10, the \"<=\" command was not allowed!"))

(deftest allowed-commands-compare-10-5
  (is      (some #(= "10 is >= 5" %) (allowed-commands compare-10-5))
            "When comparing 10 and 10, the \">=\" command was not allowed!")
  (is (not (some #(= "10 is =  5" %) (allowed-commands compare-10-5)))
            "When comparing 10 and 10, the \"=\" command was incorrectly allowed!")
  (is (not (some #(= "10 is <= 5" %) (allowed-commands compare-10-5)))
            "When comparing 10 and 10, the \"<=\" command was incorrectly allowed!"))

(deftest guarded-if-compare-10-10
  (is (some #(= "10 is >= 10" %) (times 100 '(guarded-if (nth compare-10-10 0) (nth compare-10-10 1) (nth compare-10-10 2)))))
  (is (some #(= "10 is =  10" %) (times 100 '(guarded-if (nth compare-10-10 0) (nth compare-10-10 1) (nth compare-10-10 2)))))
  (is (some #(= "10 is <= 10" %) (times 100 '(guarded-if (nth compare-10-10 0) (nth compare-10-10 1) (nth compare-10-10 2))))))

;; There are not tests for the `do` constructs.
;; It will only be visually inspected, not tested.

(deftest gcd-test
  (is 5 (gcd 10 25))
  (is 1 (gcd 1024 625))
  (is 100 (gcd 100 10000)))


;; If we define `test-ns-hook`, it is called when running `run-tests`,
;; instead of just calling all tests in the namespace.
;; This lets us control the order of the tests.
(deftest test-ns-hook
  (allowed-commands-compare-10-10)
  (allowed-commands-compare-5-10)
  (allowed-commands-compare-10-5)
  (guarded-if-compare-10-10)
  (gcd-test))

(run-tests 'testing)
\end{minted}

\subsection*{The Docker setup}
\label{sec:orgd61b11d}
The contents of the Docker setup files and scripts are repeated here.

\href{./testing/h9/Dockerfile}{Dockerfile}
\begin{minted}[breaklines=true]{docker}
# Define the argument for openjdk version
ARG OPENJDK_TAG=8u232

FROM clojure:openjdk-8
    
# Set the name of the maintainers
MAINTAINER Habib Ghaffari Hadigheh, Mark Armstrong <ghaffh1@mcmaster.ca, armstmp@mcmaster.ca>

# Set the working directory
WORKDIR /opt/h9
\end{minted}

\href{./testing/h9/docker-compose.yml}{docker-compose.yml}
\begin{minted}[breaklines=true]{yaml}
version: '2'
services:
  service:
    build: .
    image: 3mi3_h9_docker_image
    volumes:
      - .:/opt/h9
    container_name: 3mi3_h9_container
    command: bash -c
      "cat h9t.clj | lein repl"
\end{minted}

\href{./testing/h9/setup.sh}{setup.sh}
\begin{minted}[breaklines=true]{shell}
docker-compose build --force-rm
\end{minted}

\href{./testing/h9/run.sh}{run.sh}
\begin{minted}[breaklines=true]{shell}
# Run the container
docker-compose up --force-recreate
# Stop the container after finishing the test run
docker-compose stop -t 1
\end{minted}

\href{./testing/h9/README.md}{README.md}
\begin{minted}[breaklines=true]{text}
# Instructions for automated testing using Docker

We have already created a `Dockerfile` here which specifies
all the necessary packages, etc., for compiling and running your code.
You only need to follow the instructions below to see 
the results of unit tests designed to check your implementation.

## Setup
We use `docker-compose` and its configuration file to build the image.
Assuming you have `docker` and `docker-compose` installed,
simply execute
```shell script
./setup.sh
```
to generate the image.

## Prepare your code for the running the tests
You only need to place the `h8t.clj` unit test file and
the `run.sh` file in the same directory as your `h9.clj` source file.

## Running the tests
As with the build process, we have already put
the configuration needed for running the test inside `docker-compose.yml`.
Simply execute
```shell script
./run.sh
```
to run and see the results of the tests.
\end{minted}
\end{document}
