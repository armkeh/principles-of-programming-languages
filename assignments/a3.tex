% Created 2020-12-02 Wed 00:01
% Intended LaTeX compiler: lualatex
\documentclass[11pt]{article}
\usepackage{graphicx}
\usepackage{grffile}
\usepackage{longtable}
\usepackage{wrapfig}
\usepackage{rotating}
\usepackage[normalem]{ulem}
\usepackage{amsmath}
\usepackage{textcomp}
\usepackage{amssymb}
\usepackage{capt-of}
\usepackage{hyperref}
\usepackage{tabularx}
\usepackage{etoolbox}
\makeatletter
\def\dontdofcolorbox{\renewcommand\fcolorbox[4][]{##4}}
\AtBeginEnvironment{minted}{\dontdofcolorbox}
\makeatother
\usepackage[newfloat]{minted}
\usepackage{amsthm}
\theoremstyle{definition}
\newtheorem{definition}{Definition}[section]
\usepackage{unicode-math}
\usepackage{unicode}
\author{Mark Armstrong}
\date{December 2, 2020}
\title{Computer Science 3MI3 – 2020 assignment 3\\\medskip
\large A representation of Dijkstra's guarded command language}
\hypersetup{
   pdfauthor={Mark Armstrong},
   pdftitle={Computer Science 3MI3 – 2020 assignment 3},
   pdfkeywords={},
   pdfsubject={Implementing various operational semantics of various versions of the guarded command language.},
   pdfcreator={Emacs 27.0.91 (Org mode 9.4)},
   pdflang={English},
   colorlinks,
   linkcolor=blue,
   citecolor=blue,
   urlcolor=blue
   }
\begin{document}

\maketitle
\tableofcontents


\section*{Introduction}
\label{sec:org218ac0e}
This assignment 

\section*{Updates and file history}
\label{sec:org5c3a470}
\subsection*{December 1st}
\label{sec:org563183c}
\begin{itemize}
\item Initial version posted.
\end{itemize}

\section*{Boilerplate}
\label{sec:org80be87e}
\subsection*{Documentation}
\label{sec:org0bac481}
In addition to the code for the assignments,
you are required to submit (relatively light) documentation,
along the lines of that found in
\href{https://armkeh.github.io/principles-of-programming-languages/\#outline-container-Lecture-literate-programs}{the literate programs}
from lectures and tutorials.
\begin{itemize}
\item Those occasionally include a lot of writing when introducing concepts;
you do not have to introduce concepts, so your documentation
should be similar to the \emph{end} of those documents,
where only the purpose and implementation details
of types, functions, etc., are discussed.
\end{itemize}

This documentation is not assigned its own marks;
rather, 20\% of the marks of each part of the assignment
will be for the documentation.

This documentation \textbf{must be} in the literate style,
with (nicely typeset) English paragraphs alongside code snippets;
comments in your source code do not count.
The basic requirement is
\begin{itemize}
\item the English paragraphs must use non-fixed width font, whereas
\item the code snippets must use fixed width font.
\item For example, see these lecture notes on Prolog:
\begin{itemize}
\item \url{https://courses.cs.washington.edu/courses/cse341/98sp/logic/prolog.html}
\end{itemize}
\end{itemize}
But you are encouraged to strive for nicer than just
“the basic requirement”.
(the ability to write decent looking documentation is an asset!

You are free to present your documentation in any of these formats:
\begin{itemize}
\item an HTML file,
\begin{itemize}
\item (named \texttt{README.html})
\end{itemize}
\item a PDF (for instance, by writing it in \LaTeX{} using
the \texttt{listings} or \texttt{minted} package for your code blocks),
\begin{itemize}
\item (named \texttt{README.pdf}), or
\end{itemize}
\item rendering on GitLab (for instance, by writing it in markdown or Org)
\begin{itemize}
\item (named \texttt{README.md} or \texttt{README.org}.)
\end{itemize}
\end{itemize}
If you wish to use another format, contact Mark to discuss it.

Not all of your code needs to be shown;
only portions which are of interest are needed.
Feel free to omit some “repetitive” portions.
(For instance, if there are several cases in a definition
which look almost identical, only one or two need to be shown.)

\subsection*{Submission procedures}
\label{sec:orgf1e99c1}
The same guidelines as for homework
(which can be seen in any of the homework files)
apply to assignments, except for the differences below.

\subsubsection*{Assignment naming requirements}
\label{sec:org6fd5192}

Place all files for the assignment
inside a folder titled \texttt{an}, where \texttt{n} is the number of the assignment.
So, for assignment 1, use the folder \texttt{a1}, for assignment 2 the folder \texttt{a2}, etc.
Ensure you do not capitalise the \texttt{a}.

Each part of the assignments will direct you on where to
save your code for that part. Follow those instructions!

\begin{center}
\textbf{If the language supports multiple different file extensions,}
\textbf{you must still follow the extension conventions noted in the assignment.}
\end{center}

\begin{center}
\textbf{Incorrect naming of files may result in up to a 5\% deduction in your grade.}
\end{center}
This is slightly decreased from the 10\% for homeworks.

\subsection*{Proper conduct for coursework}
\label{sec:orgf0f1398}
Refer to the homework code of conduct available in any of the homework files.
The same guidelines apply to assignments.

\section*{Part 0 – The guarded command language, \emph{GCL}}
\label{sec:orgb6e7e5a}
This assignment involves representing a simple
kind of \emph{guarded command language}, which we call \emph{GCL},
and a small extension to it which we call \emph{GCLe},
which adds a notion of scope.

\subsection*{\emph{GCL}}
\label{sec:orge47e448}
The syntax of \emph{GCL} is given as
\begin{minted}[breaklines=true]{text}
⟨expr⟩ ∷= constant_integer | variable
        | ⟨expr⟩ ('+' | '*' | '-' | '÷') ⟨expr⟩

⟨test⟩ ∷= ⟨expr⟩ ('=' | '<' | '>') ⟨expr⟩
        | ⟨test⟩ ('and' | 'or') ⟨test⟩


⟨stmt⟩ ∷= 'skip'
        | variable '≔' ⟨expr⟩
        | ⟨stmt⟩ ';' ⟨stmt⟩
        | 'if' ⟨gc_list⟩
        | 'do' ⟨gc_list⟩

⟨gc⟩ ∷= ⟨test⟩ '⇒' ⟨stmt⟩

⟨gc_list⟩ ∷= { ⟨gc⟩ }
\end{minted}
That is, the language consists of
\begin{itemize}
\item (integer) expressions built from integer constants, variable names,
and the binary operations addition, multiplication, subtraction and division.
\item (boolean) tests built from equality and inequality checks on expressions,
along with \texttt{and} and \texttt{or}.
\item statements, which may be
\begin{itemize}
\item \texttt{skip}, the empty statement that does nothing,
\item assignment of an expression to a variable,
\item the composition of two statements,
\item the “choice” construct \texttt{if} applied to a list of guarded commands,
\item the “iteration” construct \texttt{do} applied to a list of guarded commands,
\end{itemize}
\item and guarded command lists, which are
a sequence of zero or more guarded commands,
\begin{itemize}
\item where a guarded command consists of a (boolean) test and a statement.
\end{itemize}
\end{itemize}

For this language, we use the same notion of (memory) state as in
the beginning of the notes on the \emph{WHILE} language:
a map or function from variable names to integers.
\textbf{We assume for this language that variables are always initialised to 0}.

The semantics of the expressions, tests and
the \texttt{skip}, assignment and composition statements
are intended to be similar to those of \emph{WHILE} as described in lecture.

The semantics of the \texttt{if} and \texttt{do} constructs on guarded command lists
are as noted in homework 9, which discussed the guarded command.
One important note: in both cases, if the guarded command list
is empty, the result should be to “do nothing”.

\subsection*{\emph{GCLe}}
\label{sec:org4ec9d90}
The language \emph{GCLe} is obtained from \emph{GCL} by adding these productions
to grammer.
\begin{minted}[breaklines=true]{text}
⟨program⟩ ∷= ⟨globals⟩ ⟨stmt⟩

⟨globals⟩ ∷= 'global' { variable }

⟨stmt⟩ ∷= 'local' variable 'in' ⟨stmt⟩
\end{minted}

The intent is that a \emph{program} now consists of
a list of global variables followed by a statement,
which we may call the “body” of the program.

Additionally, we add a new kind of statement for declaring
local variables.

With these constructs in place, we may now discuss
whether a given program is \emph{well-scoped};
that is, if every variable used in the program is either
\begin{itemize}
\item a global variable, or
\item a local variable declared by some wrapping \texttt{local} statement.
\end{itemize}

We will assume in the semantics that all programs are well-scoped,
and we can make use of a more precise notion of memory state;
a memory state is some mapping from \emph{variables which are in scope} to
values. Variables which are not in scope are not handled
by such a memory state.

\section*{Part 1 – Representations of \emph{GCL} and a small extension}
\label{sec:orgac49491}
In Ruby and in Clojure, create a representation of
the language \emph{GCL} described in part 0.

In Ruby, define the types \texttt{GCExpr}, \texttt{GCTest} and \texttt{GCStmt},
with the following subclasses.
\begin{itemize}
\item \texttt{GCExpr} has subclasses
\begin{itemize}
\item \texttt{GCConst}, the constructor of which takes a single integer argument,
\item \texttt{GCVar}, the constructor of which takes a symbol for the variable name,
\item \texttt{GCOp}, the constructor of which has as its first two arguments are \texttt{GCExpr}'s
and as its third argument a symbol,
which is intended to be one of \texttt{:plus}, \texttt{:times}, \texttt{:minus} or \texttt{:div}.
\end{itemize}
\item \texttt{GCTest} has subclasses
\begin{itemize}
\item \texttt{GCComp}, the constructor of which has as its first two arguments \texttt{GCExpr}'s
and as its third argument a symbol,
which is intended to be one of \texttt{:eq}, \texttt{:less} or \texttt{:greater},
\item \texttt{GCAnd} and \texttt{GCOr}, the constructors of which take as arguments two \texttt{GCExpr}'s.
\end{itemize}
\item \texttt{GCStmt} has subclasses
\begin{itemize}
\item \texttt{GCSkip}, the constructor of which (if it exists) takes no arguments.
\item \texttt{GCAssign}, the constructor of which takes as arguments
a symbol for the variable name and a \texttt{GCExpr}.
\item \texttt{GCCompose}, the constructor of which takes two \texttt{GCStmt}'s as arguments,
\item \texttt{GCIf} and \texttt{GCDo}, the constructors of which
take a list of \texttt{GCTest} and \texttt{GCStmt} pairs
(pairs being lists of two elements.)
\end{itemize}
\end{itemize}

Wrap all of these definitions inside a \texttt{module} named \texttt{GCL}.
(This is to avoid name clashes with definitions requested below.)

In Clojure, define \emph{records} (documentation and examples
\href{https://clojuredocs.org/clojure.core/defrecord}{here}) for each kind of expression, test and statement
(using the same naming as in Ruby.)
There is no need to define the \texttt{GCExpr}, \texttt{GCTest} and \texttt{GCStmt} types themselves;
only the subtypes as records.

Then, in Ruby, create a separate representation of
the language \emph{GCLe} described in part 0.
Create a class \texttt{GCProgram} to represent programs,
the constructor of which takes as its first argument
a list of symbols for the global variable names,
and as its second argument a \texttt{GCStmt}.
Also add an additional subclass to \texttt{GCStmt}, \texttt{GCLocal},
the constructor of which takes as its first argument a symbol
for the variable name and as its second argument a \texttt{GCStmt}.
Wrap all of these definitions inside a \texttt{module} named \texttt{GCLe}.

\section*{Part 2 – A stack machine for \emph{GCL} in Ruby}
\label{sec:org58d3b21}
Within the \texttt{GCL} module, define a method \texttt{stack\_eval} on \texttt{GCL}'s,
which carries out the evaluation of a \texttt{GCStmt} using a stack machine.

The stack machine in question should really be a method
taking three arguments:
\begin{enumerate}
\item the command stack (implemented using a list),
\item the results stack (implemented a list), and
\item the memory state (implemented using a lambda; that is, a block.)
\end{enumerate}

The method should return an updated state
(that is, another lambda/block.)

\section*{Part 3 – The small-step semantics of \emph{GCL} in Clojure}
\label{sec:orgad4ee73}
Define in Clojure a function \texttt{reduce} which takes
a \emph{GCL} statement and a memory state
(a function mapping symbols, representing the variable names, to integers)
and performs \emph{one step} of the computation, returning the
remaining code to be run and the updated memory state.

\section*{Part 4 – The big-step semantics of \emph{GCLe} in Ruby}
\label{sec:orgc180894}
This portion of the assignment should be done in
the \texttt{GCLe} module created in part 1.

Begin by defining a method \texttt{wellScoped} which checks that
all variables appearing within the body of a \texttt{GCProgram}
(either in an expression or on the left side of an assignment)
are \emph{within scope} at the point of their use;
that is, either the variable is one declared to be \texttt{global},
or there is a \texttt{local} statement for that variable wrapping the use.
\begin{itemize}
\item This method should take a \texttt{GCStmt} as its only argument,
and return a boolean.
\item Hint: This operation is similar to typechecking.
Use your experience working with \texttt{typeOf} as a starting point.
\begin{itemize}
\item Helper methods are always permitted.
\end{itemize}
\end{itemize}

Then define the semantics of the language,
this time defining a method \texttt{eval} directly
(without making use of a stack machine.)
That is, define the \emph{big-step} semantics of the language
(remember that big-step semantics are called evaluation semantics.)
\begin{itemize}
\item This method also should take a \texttt{GCStmt} as its only argument.
It should return a \texttt{Hash} mapping the \texttt{global} variable names
to integers.
\end{itemize}

You may decide what the behaviour is for programs which
do not initialise variables before their first use.
\begin{itemize}
\item Your choice may be judged in the marking.
\begin{itemize}
\item It is suggested that such programs “fail gracefully”,
reporting an error that a variable was used before initialisation.
\item Otherwise, it's suggested that they behave as predictably as possible.
\end{itemize}
\end{itemize}

\section*{Part 5 – \emph{GCLe} in Clojure}
\label{sec:org8220426}
\emph{As a bonus}, repeat part 4 in Clojure.

Place the code for this portion in a file \texttt{a3b.clj}.

This time, you may choose the underlying approach to the operational semantics
(you do not have to use big-step semantics.)

Document this portion especially well, and include your own
tests in a file \texttt{a3bt.clj}. This file should output the results of the tests
when executed using \texttt{cat a3bt.clj | lein} from the command line.

\section*{Submission checklist}
\label{sec:orgb723f35}
For your convenience, this checklist is provided
to track the files you need to submit.
Use it if you wish.
\begin{minted}[breaklines=true]{text}
- [ ] Documentation; one of
  - [ ] README.html
  - [ ] README.pdf
  - [ ] README.md
  - [ ] README.org
- [ ] Code files
  - [ ] a3.rb
  - [ ] a3.clj
- [ ] Part 2 tests
  - [ ] a3p2_test.rb tests have passed! (No submission needed.)
- [ ] Part 3 tests
  - [ ] a3p3_test.clj tests have passed! (No submission needed.)
- [ ] Part 4 tests
  - [ ] a3p4_test.clj tests have passed! (No submission needed.)
- [ ] Part 5 (Bonus)
  - [ ] a3b.clj
  - [ ] a3bt.clj
\end{minted}

\section*{Testing}
\label{sec:org65c579b}
:TODO:
\end{document}
