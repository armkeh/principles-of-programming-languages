% Created 2020-11-09 Mon 11:22
% Intended LaTeX compiler: lualatex
\documentclass[11pt]{article}
\usepackage{graphicx}
\usepackage{grffile}
\usepackage{longtable}
\usepackage{wrapfig}
\usepackage{rotating}
\usepackage[normalem]{ulem}
\usepackage{amsmath}
\usepackage{textcomp}
\usepackage{amssymb}
\usepackage{capt-of}
\usepackage{hyperref}
\usepackage{tabularx}
\usepackage{etoolbox}
\makeatletter
\def\dontdofcolorbox{\renewcommand\fcolorbox[4][]{##4}}
\AtBeginEnvironment{minted}{\dontdofcolorbox}
\makeatother
\usepackage[newfloat]{minted}
\usepackage{amsthm}
\theoremstyle{definition}
\newtheorem{definition}{Definition}[section]
\usepackage{unicode-math}
\usepackage{unicode}
\author{Mark Armstrong}
\date{Fall 2020}
\title{A typed λ-calculus, \emph{TL}\\\medskip
\large Principles of Programming Languages}
\hypersetup{
   pdfauthor={Mark Armstrong},
   pdftitle={A typed λ-calculus, \emph{TL}},
   pdfkeywords={},
   pdfsubject={Extending our lambda calculus with type checking (enforcement).},
   pdfcreator={Emacs 27.0.91 (Org mode 9.4)},
   pdflang={English},
   colorlinks,
   linkcolor=blue,
   citecolor=blue,
   urlcolor=blue
   }
\begin{document}

\maketitle

\section{Preamble}
\label{sec:orgf04e6a5}

\subsection{Notable references}
\label{sec:org0f80e78}

\begin{itemize}
\item Benjamin Pierce,
“\href{https://ebookcentral.proquest.com/lib/mcmu/detail.action?docID=3338823}{Types and Programming Languages}”
\begin{itemize}
\item Chapter 9, Simply Typed Lambda-Calculus
\begin{itemize}
\item Function types, the typing relation
\end{itemize}
\item Chapter 11, Simple Extensions
\begin{itemize}
\item Unit, Tuples, Sums, Variants, Lists.
\end{itemize}
\end{itemize}
\end{itemize}

\subsection{{\bfseries\sffamily TODO} Table of contents}
\label{sec:org12e2d77}

\begin{scriptsize}
\begin{itemize}
\item \hyperref[sec:orgf04e6a5]{Preamble}
\end{itemize}
\end{scriptsize}

\section{Introduction}
\label{sec:org33744d9}

In this section we extend our previously considered
untyped λ-calculus by defining a typing relation,
essentially adding type checking (enforcement).

We then investigate adding some
algebraic type formers to the language.
This involves the introduction of a rudimentary form
of pattern matching.

\section{Recall: The untyped λ-calculus}
\label{sec:org611891e}

Recall from section 3 of the notes the syntax
of our untyped λ-calculus, \emph{UL}.
\begin{minted}[breaklines=true]{text}
⟨term⟩ ∷= var | λ var → ⟨term⟩ | ⟨term⟩ ⟨term⟩
\end{minted}

Recall that in this pure untyped λ-calculus,
everything is a function, and abstractions
(terms of the form \texttt{λ x → t}) are \emph{values}.

\subsection{The call-by-value semantics of the untyped λ-calculus}
\label{sec:org8344f43}

The call-by-value semantics we described in section 3 of the notes
can be more succinctly described using inference rules.
\begin{itemize}
\item In fact, we only need three rules.

\item Here the arrow \texttt{⟶} defines a \emph{reduction} relation,
meaning that we may need to perform several \texttt{⟶} “steps” to fully
evaluate a term.
\item The (meta)variables \texttt{t₁}, \texttt{t₂}, etc., range over λ-terms, and
\item the (meta)variables \texttt{v₁}, \texttt{v₂}, etc., range over λ-terms \emph{which are values}.
\end{itemize}
\begin{minted}[breaklines=true]{text}
  t₁  ⟶  t₁′
––––––––––––––––––– reduce-appˡ
 t₁ t₂  ⟶  t₁′ t₂

 
  t₂  ⟶  t₂′
––––––––––––––––––– reduce-Appʳ
 v₁ t₂  ⟶  v₁ t₂′


–––––––––––––––––––––––––––––– apply
(λ x → t) v  ⟶  t[x ≔ v]
\end{minted}

\subsection{Only applications reduce}
\label{sec:org9da3ba7}

\iffalse
\begin{minted}[breaklines=true]{text}
  t₁  ⟶  t₁′                             t₂  ⟶  t₂′                   
––––––––––––––––––– reduce-appˡ        ––––––––––––––––––– reduce-Appʳ
 t₁ t₂  ⟶  t₁′ t₂                       v₁ t₂  ⟶  v₁ t₂′              


 
             –––––––––––––––––––––––––––––– apply
             (λ x → t) v  ⟶  t[x ≔ v]
\end{minted}
\fi

Notice, in the above semantics, that the only rules
are for applications; remember that
\begin{itemize}
\item variables cannot be reduced, and
\item under call-by-value semantics,
\begin{itemize}
\item no evaluations take place inside abstractions, and
\item abstractions are only applied to values.
\end{itemize}
\end{itemize}

\subsection{Explaining the rules}
\label{sec:org3768f1d}

\iffalse
\begin{minted}[breaklines=true]{text}
  t₁  ⟶  t₁′                             t₂  ⟶  t₂′                   
––––––––––––––––––– reduce-appˡ        ––––––––––––––––––– reduce-Appʳ
 t₁ t₂  ⟶  t₁′ t₂                       v₁ t₂  ⟶  v₁ t₂′              


 
             –––––––––––––––––––––––––––––– apply
             (λ x → t) v  ⟶  t[x ≔ v]
\end{minted}
\fi

By using our naming conventions, we can see that
\begin{itemize}
\item the \texttt{reduce-appˡ} rule says that
if \texttt{t₁} is the left side of an application
and \texttt{t₁} reduces to \texttt{t₁′}, then the whole application reduces
by replacing \texttt{t₁} with \texttt{t₁′},
\item the \texttt{reduce-appʳ} rule says that
if \texttt{t₁} is the right side of
an application \emph{whose left side is a value},
and \texttt{t₂} reduces to \texttt{t₂′}, then the whole application reduces
by replacing \texttt{t₂} with \texttt{t₂′}, and
\item the \texttt{apply} rule says that if the left side of an application
is an abstraction, and the right side is a value,
then the application reduces to the body of the abstraction
with the value substituted for the abstraction's variable.
\end{itemize}

\subsection{Reduction as a function}
\label{sec:org193b251}

\iffalse
\begin{minted}[breaklines=true]{text}
  t₁  ⟶  t₁′                             t₂  ⟶  t₂′                   
––––––––––––––––––– reduce-appˡ        ––––––––––––––––––– reduce-Appʳ
 t₁ t₂  ⟶  t₁′ t₂                       v₁ t₂  ⟶  v₁ t₂′              


 
             –––––––––––––––––––––––––––––– apply
             (λ x → t) v  ⟶  t[x ≔ v]
\end{minted}
\fi

It bears noting that the \emph{reduction relation} here is,
by design, \emph{deterministic}; given a λ-term \texttt{t}, either
\begin{itemize}
\item \texttt{t} can be reduced by exactly \emph{one} of the rules above, or
\item \texttt{t} cannot be reduced (is irreducible) (by these semantics.)
\end{itemize}

A deterministic relation can be expressed as a \emph{function},
as the following Scala-like pseudocode shows.
\begin{verbatim}
def ⟶(t) = t match {
  case t₁ t₂ if t₁ ⟶ t₁′                 => t₁′ t₂
  case v₁ t₂ if isValue(v₁) &&  t₂ ⟶ t₂′ => v₁ t₂′
  case (λ x → t) v if isValue(v)         => t[x ≔ v]
} 
\end{verbatim}

\subsection{An example of a reduction sequence}
\label{sec:org1aa31f0}

\begin{minted}[breaklines=true]{text}
  ((λ x → x) (λ y → y)) ((λ z → z) (λ u → u))
⟶⟨ reduce-appˡ ⟩
  (λ y → y) ((λ z → z) (λ u → u))
⟶⟨ reduce-appʳ ⟩
  (λ y → y) (λ u → u)
⟶⟨ apply ⟩
  λ u → u
\end{minted}
The final term does not reduce.

Note that we can end with terms which do not reduce, but which
are not values, such as
\begin{minted}[breaklines=true]{text}
(λ x → x) y
\end{minted}
Since free variables are not values (they are not λ-abstractions),
this term does not fit any of the reduction rules.

\subsection{Encodings of booleans, natural numbers and pairs}
\label{sec:orgcfafcb2}

Recall the λ-encodings discussed in notes section 3,
which allow us to represent booleans, natural numbers
and pairs in the pure untyped λ-calculus.
\begin{minted}[breaklines=true]{text}
tru  = λ t → λ f → t
fls  = λ t → λ f → f
test = λ l → λ m → λ n → l m n
pair = λ f → λ s → λ b → b f s
fst  = λ p → p tru
snd  = λ p → p fls
zero = λ s → λ z → z
scc  = λ n → λ s → λ z → s (n s z)
\end{minted}

\subsection{Enriching the (syntax of the) calculus}
\label{sec:orgb583b0b}

While λ-encodings of data in the pure untyped λ-calculus,
such as those for the booleans, natural numbers and pairs,
do allow us to construct programs working on any type data
we might like, it is usually more convenient
(even in this untyped system)
to instead \emph{enrich} the calculus with new primitive terms
for the types we want to work with.

We will show here how this can be done for booleans.
The enriched calculus's syntax is then
\begin{minted}[breaklines=true]{text}
⟨term⟩ ∷= var | λ var → ⟨term⟩ | ⟨term⟩ ⟨term⟩
        | true | false
        | if ⟨term⟩ then ⟨term⟩ else ⟨term⟩
\end{minted}

\subsection{The semantics of the extended calculus}
\label{sec:orgca45f25}

:TODO:

\section{The simply typed λ-calculus}
\label{sec:org55bb4cc}

:TODO:

\section{“Simple extensions” to the simply typed λ-calculus}
\label{sec:org1d8a0ab}

:TODO:
\end{document}
