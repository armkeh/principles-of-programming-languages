% Created 2020-11-16 Mon 11:33
% Intended LaTeX compiler: lualatex
\documentclass[11pt]{article}
\usepackage{graphicx}
\usepackage{grffile}
\usepackage{longtable}
\usepackage{wrapfig}
\usepackage{rotating}
\usepackage[normalem]{ulem}
\usepackage{amsmath}
\usepackage{textcomp}
\usepackage{amssymb}
\usepackage{capt-of}
\usepackage{hyperref}
\usepackage{tabularx}
\usepackage{etoolbox}
\makeatletter
\def\dontdofcolorbox{\renewcommand\fcolorbox[4][]{##4}}
\AtBeginEnvironment{minted}{\dontdofcolorbox}
\makeatother
\usepackage[newfloat]{minted}
\usepackage{amsthm}
\theoremstyle{definition}
\newtheorem{definition}{Definition}[section]
\usepackage{unicode-math}
\usepackage{unicode}
\author{Mark Armstrong}
\date{Fall 2020}
\title{A typed λ-calculus, \emph{TL}\\\medskip
\large Principles of Programming Languages}
\hypersetup{
   pdfauthor={Mark Armstrong},
   pdftitle={A typed λ-calculus, \emph{TL}},
   pdfkeywords={},
   pdfsubject={Extending our lambda calculus with type checking (enforcement).},
   pdfcreator={Emacs 27.0.91 (Org mode 9.4)},
   pdflang={English},
   colorlinks,
   linkcolor=blue,
   citecolor=blue,
   urlcolor=blue
   }
\begin{document}

\maketitle

\section{Preamble}
\label{sec:org89631f5}

\subsection{Notable references}
\label{sec:org8279d70}

\begin{itemize}
\item Benjamin Pierce,
“\href{https://ebookcentral.proquest.com/lib/mcmu/detail.action?docID=3338823}{Types and Programming Languages}”
\begin{itemize}
\item Chapter 9, Simply Typed Lambda-Calculus
\begin{itemize}
\item Function types, the typing relation
\end{itemize}
\item Chapter 11, Simple Extensions
\begin{itemize}
\item Unit, Tuples, Sums, Variants, Lists.
\end{itemize}
\end{itemize}
\end{itemize}

\subsection{{\bfseries\sffamily TODO} Table of contents}
\label{sec:orgca77fdf}

\begin{scriptsize}
\begin{itemize}
\item \hyperref[sec:org89631f5]{Preamble}
\end{itemize}
\end{scriptsize}

\section{Introduction}
\label{sec:org658b9c8}

In this section we extend our previously considered
untyped λ-calculus by defining a typing relation,
essentially adding type checking (enforcement).

We then investigate adding some
algebraic type formers to the language.
This involves the introduction of a rudimentary form
of pattern matching.

\section{Recall: The untyped λ-calculus}
\label{sec:orgb6da187}

Recall from section 3 of the notes the syntax
of our untyped λ-calculus, \emph{UL}.
\begin{minted}[breaklines=true]{text}
⟨term⟩ ∷= var | λ var → ⟨term⟩ | ⟨term⟩ ⟨term⟩
\end{minted}

Recall that in this pure untyped λ-calculus,
everything is a function, and abstractions
(terms of the form \texttt{λ x → t}) are \emph{values}.

\subsection{The call-by-value semantics of the untyped λ-calculus}
\label{sec:org653b774}

The call-by-value semantics we described in section 3 of the notes
can be more succinctly described using inference rules.
\begin{itemize}
\item In fact, we only need three rules.

\item Here the arrow \texttt{⟶} defines a \emph{reduction} relation,
meaning that we may need to perform several \texttt{⟶} “steps” to fully
evaluate a term.
\item The (meta)variables \texttt{t₁}, \texttt{t₂}, etc., range over λ-terms, and
\item the (meta)variables \texttt{v₁}, \texttt{v₂}, etc., range over λ-terms \emph{which are values}.
\end{itemize}
\begin{minted}[breaklines=true]{text}
  t₁  ⟶  t₁′
––––––––––––––––––– reduce-appˡ
 t₁ t₂  ⟶  t₁′ t₂

 
  t₂  ⟶  t₂′
––––––––––––––––––– reduce-appʳ
 v₁ t₂  ⟶  v₁ t₂′


–––––––––––––––––––––––––––––– apply
(λ x → t) v  ⟶  t[x ≔ v]
\end{minted}

\subsection{Only applications reduce}
\label{sec:org9d6e666}

\iffalse
\begin{minted}[breaklines=true]{text}
  t₁  ⟶  t₁′                             t₂  ⟶  t₂′                   
––––––––––––––––––– reduce-appˡ        ––––––––––––––––––– reduce-appʳ
 t₁ t₂  ⟶  t₁′ t₂                       v₁ t₂  ⟶  v₁ t₂′              


 
             –––––––––––––––––––––––––––––– apply
             (λ x → t) v  ⟶  t[x ≔ v]
\end{minted}
\fi

Notice, in the above semantics, that the only rules
are for applications; remember that
\begin{itemize}
\item variables cannot be reduced, and
\item under call-by-value semantics,
\begin{itemize}
\item no evaluations take place inside abstractions, and
\item abstractions are only applied to values.
\end{itemize}
\end{itemize}

\subsection{Explaining the rules}
\label{sec:orgd9adc94}

\iffalse
\begin{minted}[breaklines=true]{text}
  t₁  ⟶  t₁′                             t₂  ⟶  t₂′                   
––––––––––––––––––– reduce-appˡ        ––––––––––––––––––– reduce-appʳ
 t₁ t₂  ⟶  t₁′ t₂                       v₁ t₂  ⟶  v₁ t₂′              


 
             –––––––––––––––––––––––––––––– apply
             (λ x → t) v  ⟶  t[x ≔ v]
\end{minted}
\fi

By using our naming conventions, we can see that
\begin{itemize}
\item the \texttt{reduce-appˡ} rule says that
if \texttt{t₁} is the left side of an application
and \texttt{t₁} reduces to \texttt{t₁′}, then the whole application reduces
by replacing \texttt{t₁} with \texttt{t₁′},
\item the \texttt{reduce-appʳ} rule says that
if \texttt{t₁} is the right side of
an application \emph{whose left side is a value},
and \texttt{t₂} reduces to \texttt{t₂′}, then the whole application reduces
by replacing \texttt{t₂} with \texttt{t₂′}, and
\item the \texttt{apply} rule says that if the left side of an application
is an abstraction, and the right side is a value,
then the application reduces to the body of the abstraction
with the value substituted for the abstraction's variable.
\end{itemize}

\subsection{Reduction as a function}
\label{sec:org8a6dd63}

\iffalse
\begin{minted}[breaklines=true]{text}
  t₁  ⟶  t₁′                             t₂  ⟶  t₂′                   
––––––––––––––––––– reduce-appˡ        ––––––––––––––––––– reduce-appʳ
 t₁ t₂  ⟶  t₁′ t₂                       v₁ t₂  ⟶  v₁ t₂′              


 
             –––––––––––––––––––––––––––––– apply
             (λ x → t) v  ⟶  t[x ≔ v]
\end{minted}
\fi

It bears noting that the \emph{reduction relation} here is,
by design, \emph{deterministic}; given a λ-term \texttt{t}, either
\begin{itemize}
\item \texttt{t} can be reduced by exactly \emph{one} of the rules above, or
\item \texttt{t} cannot be reduced (is irreducible) (by these semantics.)
\end{itemize}

A deterministic relation can be expressed as a \emph{function},
as the following Scala-like pseudocode shows.
\begin{verbatim}
def ⟶(t) = t match {
  case t₁ t₂ if t₁ ⟶ t₁′                 => t₁′ t₂
  case v₁ t₂ if isValue(v₁) &&  t₂ ⟶ t₂′ => v₁ t₂′
  case (λ x → t) v if isValue(v)         => t[x ≔ v]
} 
\end{verbatim}

\subsection{An example of a reduction sequence}
\label{sec:orgf04411f}

\begin{minted}[breaklines=true]{text}
  ((λ x → x) (λ y → y)) ((λ z → z) (λ u → u))
⟶⟨ reduce-appˡ ⟩
  (λ y → y) ((λ z → z) (λ u → u))
⟶⟨ reduce-appʳ ⟩
  (λ y → y) (λ u → u)
⟶⟨ apply ⟩
  λ u → u
\end{minted}
The final term does not reduce.

Note that we can end with terms which do not reduce, but which
are not values, such as
\begin{minted}[breaklines=true]{text}
(λ x → x) y
\end{minted}
Since free variables are not values (they are not λ-abstractions),
this term does not fit any of the reduction rules.

\subsection{Encodings of booleans, natural numbers and pairs}
\label{sec:orga48a8e8}

Recall the λ-encodings discussed in notes section 3,
which allow us to represent booleans, natural numbers
and pairs in the pure untyped λ-calculus.
\begin{minted}[breaklines=true]{text}
tru  = λ t → λ f → t
fls  = λ t → λ f → f
test = λ l → λ m → λ n → l m n
pair = λ f → λ s → λ b → b f s
fst  = λ p → p tru
snd  = λ p → p fls
zero = λ s → λ z → z
scc  = λ n → λ s → λ z → s (n s z)
\end{minted}

\subsection{Enriching the (syntax of the) calculus}
\label{sec:org7cb689f}

While λ-encodings of data in the pure untyped λ-calculus,
such as those for the booleans, natural numbers and pairs,
do allow us to construct programs working on any type data
we might like, it is usually more convenient
(even in this untyped system)
to instead \emph{enrich} the calculus with new primitive terms
for the types we want to work with.

We will show here how this can be done for booleans.
The enriched calculus's syntax is then
\begin{minted}[breaklines=true]{text}
⟨term⟩ ∷= var | λ var → ⟨term⟩ | ⟨term⟩ ⟨term⟩
        | true | false
        | if ⟨term⟩ then ⟨term⟩ else ⟨term⟩
\end{minted}

\subsection{The semantics of the untyped λ-calculus with booleans}
\label{sec:org0bf7dfd}

The untyped λ-calculus extended with booleans semantics has,
in addition to the rules \texttt{reduce-appˡ}, \texttt{reduce-appʳ} and \texttt{apply},
these rules for the new basic primitive functions.
\begin{minted}[breaklines=true]{text}
          tᵇ  ⟶  tᵇ′
–––––––––––––––––––––––––––––––––– reduce-if
 if tᵇ t₁ t₂  ⟶  if tᵇ′ t₁ t₂

 

––––––––––––––––––––––––– if-then
 if true t₁ t₂  ⟶  t₁



––––––––––––––––––––––––– if-else
 if false t₁ t₂  ⟶  t₂
\end{minted}

\section{A first typed λ-calculus – the simply typed λ-calculus}
\label{sec:org03a87bf}

Starting now, we define the syntax and semantics for several stages
of a typed λ-calculus.
\begin{itemize}
\item We begin with a “simply-typed” λ-calculus that has only
unit and function types, and
\item at each stage (in the following sections of these notes),
we add new primitive terms, new types
and typing rules, and new semantic rules.
\end{itemize}

These stages roughly correspond to those given in
Pierce's “Types and Programming Languages” throughout
chapters
\begin{itemize}
\item 9, “Simply Typed Lambda-Calculus”, and
\item 11, “Simple Extensions”.
\end{itemize}

For the sake of page space, each stage will only show
the grammar productions and semantic rule which are added,
not the whole grammar or semantics.
\begin{itemize}
\item Those will be given at the end.
\end{itemize}

All semantics in this section are call-by-value semantics.

\subsection{Typing rules}
\label{sec:org8d5ad05}

Like semantics, the typing rules of a language
are presented here using inference rules.

These inference rules define a typing relation,
written \(\_⊢\_:\_\) and read as “entails”.

While the reduction relation, \(\_⟶\_\), is a binary relation between terms
\begin{itemize}
\item i.e., \(\_⟶\_ : \texttt{term} × \texttt{term}\)
\begin{itemize}
\item (in fact, since it is a single-valued relation, \(\_⟶\_ : \texttt{term} → \texttt{term}\)),
\end{itemize}
\end{itemize}
the typing relation is a \emph{ternary} relation between a \emph{typing context},
a term and a type.
\begin{itemize}
\item i.e., \(\_⊢\_:\_ \ \ : \ \ \texttt{context} × \texttt{term} × \texttt{type}\)
\begin{itemize}
\item (in fact, since it is also a single-valued relation,
\(\_⊢\_:\_ \ \ : \ \ \texttt{context} × \texttt{term} → \texttt{type}\).)
\end{itemize}
\end{itemize}

\subsection{The typing context}
\label{sec:org1dbeaac}

The \emph{typing context} referred to above is a set
of variable, type pairs, used to \emph{bind} certain variables to types.
\begin{itemize}
\item It can in fact be a sequence or similar datatype;
so long as we can add and check bindings.
\end{itemize}

We will write
\begin{itemize}
\item \texttt{∅} for the \emph{empty} typing context,
\item \texttt{Γ,(x : A)} to \emph{extend} the typing context \texttt{Γ} with the additional
type binding of \texttt{x} to \texttt{A}, and
\item \texttt{(x : A) ∈ Γ} to check if \texttt{x} is bound to type \texttt{A} by the typing context \texttt{Γ}.
\end{itemize}

\subsection{Example typing contexts}
\label{sec:orgc9eb34f}

For example,
\begin{itemize}
\item \texttt{(x : A) ∈ (∅,(z : C),(y : C),(x : A)}) and
\item \texttt{(y : C) ∈ (∅,(z : C),(y : C),(x : A)}) and
\end{itemize}
but
\begin{itemize}
\item NOT \texttt{(x : B) ∈ (∅,(z : C),(y : C),(x : A)}) and
\item NOT \texttt{(y : B) ∈ (∅,(z : C),(y : C),(x : A)}) and
\end{itemize}

We generally try avoid having two entries for a variable in the typing context
(such as \texttt{∅,(x : A),(x : B)} or even \texttt{∅,(x : A),(x : A)}.)
\begin{itemize}
\item This will occur in practice if variable names are reused.
\end{itemize}
In the case that there are two such entries, the later one
“shadows” the earlier one.
\begin{itemize}
\item So for instance, \texttt{(x : B) ∈ (∅,(x : A),(x : B))} and
\item NOT \texttt{(x : A) ∈ (∅,(x : A),(x : B))}.
\end{itemize}

\subsection{The simply-typed λ-calculus syntax}
\label{sec:orgc61c562}

Our starting point is the simply-typed λ-calculus,
which has only unit and function types.
\begin{itemize}
\item For the sake of noting which new terms are values,
we add a non-terminal called \texttt{⟨value⟩} to the grammar.
\end{itemize}
\begin{minted}[breaklines=true]{text}
⟨term⟩ ∷= var
        | ⟨term⟩ ⟨term⟩
        | ⟨value⟩
        
⟨value⟩ ∷= λ var : ⟨type⟩ → ⟨term⟩
         | unit
         
⟨type⟩ ∷= Unit | ⟨type⟩ → ⟨type⟩
\end{minted}

\subsection{The simply-typed λ-calculus typing}
\label{sec:org3f46aab}

“If a variable \texttt{x} is assigned type \texttt{A} by the context,
then it has that type.”
\begin{minted}[breaklines=true]{text}
 (x : A) ∈ Γ
––––––––––––––– T-Var
 Γ ⊢ x : A
\end{minted}
Notice that otherwise, variables do not typecheck!

“The abstraction of a variable \texttt{x} of type \texttt{A} over a term \texttt{t} has
type \texttt{A → B} if \texttt{t} has type \texttt{B} when assuming \texttt{x} has type \texttt{A}.” 
\begin{minted}[breaklines=true]{text}
 Γ,(x : A) ⊢ t : B
––––––––––––––––––––––––––––––– T-Abs
 Γ ⊢ (λ x : A → t) : A → B
\end{minted}

“If \texttt{t₁} has type \texttt{A → B} and \texttt{t₂} type \texttt{A}, then applying \texttt{t₁} to \texttt{t₂} has type \texttt{B}.”
\begin{minted}[breaklines=true]{text}
 Γ ⊢ t₁ : A → B    Γ ⊢ t₂ : A
––––––––––––––––––––––––––––––––– T-App
        Γ ⊢ t₁ t₂ : B
\end{minted}

“\texttt{unit} has type \texttt{Unit}.” 
\begin{minted}[breaklines=true]{text}
––––––––––––––––––––––––––––––––– T-Unit
        Γ ⊢ unit : Unit
\end{minted}

\subsection{The simply-typed λ-calculus semantics}
\label{sec:org30b2459}

The semantics of the language have not changed,
except that the syntax of the λ-abstraction now
has the type annotation.
\begin{minted}[breaklines=true]{text}
  t₁  ⟶  t₁′
––––––––––––––––––– reduce-appˡ
 t₁ t₂  ⟶  t₁′ t₂

 
  t₂  ⟶  t₂′
––––––––––––––––––– reduce-appʳ
 v₁ t₂  ⟶  v₁ t₂′


–––––––––––––––––––––––––––––– apply
(λ x : A → t) v  ⟶  t[x ≔ v]
\end{minted}

\subsection{Exercise: Why do we need a \texttt{Unit} type in the simply-typed λ-calculus?}
\label{sec:orgd186d24}

Recall that in the untyped λ-calculus,
the only values were abstractions; all data was functions.

Why do we add a \texttt{Unit} type in the simply-typed λ-calculus?
Is it required for some reason?

\subsection{Exercise: Type some terms}
\label{sec:org2681edc}

Using the typing rules above, determine the types of the following
simply-typed λ-calculus terms.
(You should try to give a derivation of the type using the rules.)

\begin{enumerate}
\item \texttt{(λ x : Unit → x)}.
\item \texttt{(λ x : Unit → x) (unit)}.
\item \texttt{(λ x : (Unit → Unit) → λ y : Unit → x y) (λ z : Unit → unit)}.
\item \texttt{(λ x : (Unit → Unit) → λ y : Unit → x y) (λ z : Unit → unit) (unit)}.
\end{enumerate}

For these terms, try to justify which portion(s) of the term
causes it not to typecheck.
\begin{enumerate}
\item \texttt{x}.
\item \texttt{(λ x : Unit → x) (λ x : Unit → x)}.
\end{enumerate}

\section{Adding natural numbers and booleans}
\label{sec:org1a7da2a}

We begin our extensions to the simply-typed λ-calculus
with the natural numbers and booleans,
convenient types to have for many simply computing problems.

\subsection{Syntax for natural numbers and booleans}
\label{sec:orgaba9367}

Recall that these productions are only those being added;
the productions from \ref{sec:orgc61c562} are
assumed to still be in place.

Note that natural numbers are considered values only if
they are a chain of \texttt{suc}'s applied to a (natural number) value.
\begin{itemize}
\item The “(natural number)” portion of that statement will be
enforced by the typing rules.
\item For instance, \texttt{suc zero} is a value
(as is \texttt{succ true} according to the syntax,
but the typing rules will forbid that.)
\item \texttt{succ} also appears in a production for \texttt{⟨term⟩} to allow for its use in non-values
(such as \texttt{succ pred zero}, which should simplify to \texttt{zero} under our semantics.)
\end{itemize}

\begin{minted}[breaklines=true]{text}
⟨term⟩ ∷= suc ⟨term⟩
        | pred ⟨term⟩
        | iszero ⟨term⟩
        | if ⟨term⟩ then ⟨term⟩ else ⟨term⟩

⟨value⟩ ∷= suc ⟨value⟩
         | zero
         | true
         | false
\end{minted}

\subsection{Typing for natural numbers and booleans}
\label{sec:org1e48df6}

\begin{minted}[breaklines=true]{text}
                               
––––––––––––––––––– T-zero     
 Γ ⊢ zero : nat
 
\end{minted}

\begin{minted}[breaklines=true]{text}

   Γ ⊢ t : nat                   Γ ⊢ t : nat                 
–––––––––––––––––––– T-suc    –––––––––––––––––––– T-pred                                 
 Γ ⊢ suc t : nat               Γ ⊢ pred t : nat                                           

\end{minted}

\begin{minted}[breaklines=true]{text}

––––––––––––––––––– T-true    –––––––––––––––––––– T-false
 Γ ⊢ true : bool               Γ ⊢ false : bool

\end{minted}

\begin{minted}[breaklines=true]{text}

    Γ ⊢ t : nat
–––––––––––––––––––––––– T-iszero
 Γ ⊢ iszero t : bool 

\end{minted}

\begin{minted}[breaklines=true]{text}

 Γ ⊢ b : bool    Γ ⊢ t₁ : A    Γ ⊢ t₂ : A
––––––––––––––––––––––––––––––––––––––––––––– T-if
         Γ ⊢ if b then t₁ else t₂ : A

\end{minted}

\subsection{Semantics of natural numbers and booleans}
\label{sec:org9f7f993}

\begin{minted}[breaklines=true]{text}

      t ⟶ t′                              t ⟶ t′                 
–––––––––––––––––––– reduce-suc    –––––––––––––––––––– reduce-pred                                 
  suc t ⟶ suc t′                     pred t ⟶ pred t′                                           

\end{minted}

\begin{minted}[breaklines=true]{text}

      t ⟶ t′                                              
––––––––––––––––––––––––– reduce-iszero    ––––––––––––––––––––––– iszero-zero
  iszero t ⟶ iszero t′                       iszero zero ⟶ true                                           

\end{minted}

The natural number predecessor of zero is zero.
And because of this, if we have the successor \emph{of a value},
then it's known to not be zero.
\begin{minted}[breaklines=true]{text}

                                              
–––––––––––––––––––––– pred-zero    ––––––––––––––––––––––––––––– iszero-suc
  pred zero ⟶ zero                    iszero (suc v) ⟶ false                                           

\end{minted}

\begin{minted}[breaklines=true]{text}

 iszero t ⟶ false
–––––––––––––––––––– suc-pred    –––––––––––––––––––– pred-suc
 suc (pred t) ⟶ t                  pred (suc t) ⟶ t                                           

\end{minted}


\iffalse

\subsection{Semantics of natural numbers and booleans continued}
\label{sec:orgeedd2fd}

\fi

\begin{minted}[breaklines=true]{text}

          tᵇ  ⟶  tᵇ′
–––––––––––––––––––––––––––––––––– reduce-if
 if tᵇ t₁ t₂  ⟶  if tᵇ′ t₁ t₂

\end{minted}

\begin{minted}[breaklines=true]{text}

––––––––––––––––––––––––– if-then
 if true t₁ t₂  ⟶  t₁

\end{minted}

\begin{minted}[breaklines=true]{text}

––––––––––––––––––––––––– if-else
 if false t₁ t₂  ⟶  t₂
\end{minted}
\end{document}
