% Created 2020-10-26 Mon 11:15
% Intended LaTeX compiler: lualatex
\documentclass[11pt]{article}
\usepackage{graphicx}
\usepackage{grffile}
\usepackage{longtable}
\usepackage{wrapfig}
\usepackage{rotating}
\usepackage[normalem]{ulem}
\usepackage{amsmath}
\usepackage{textcomp}
\usepackage{amssymb}
\usepackage{capt-of}
\usepackage{hyperref}
\usepackage{tabularx}
\usepackage{etoolbox}
\makeatletter
\def\dontdofcolorbox{\renewcommand\fcolorbox[4][]{##4}}
\AtBeginEnvironment{minted}{\dontdofcolorbox}
\makeatother
\usepackage[newfloat]{minted}
\usepackage{amsthm}
\theoremstyle{definition}
\newtheorem{definition}{Definition}[section]
\usepackage{unicode-math}
\usepackage{unicode}
\author{Mark Armstrong}
\date{Fall 2020}
\title{Types\\\medskip
\large Principles of Programming Languages}
\hypersetup{
   pdfauthor={Mark Armstrong},
   pdftitle={Types},
   pdfkeywords={},
   pdfsubject={Introduction to types},
   pdfcreator={Emacs 27.0.90 (Org mode 9.4)},
   pdflang={English},
   colorlinks,
   linkcolor=blue,
   citecolor=blue,
   urlcolor=blue
   }
\begin{document}

\maketitle

\section{Preamble}
\label{sec:org5f263db}

\subsection{{\bfseries\sffamily TODO} Notable references}
\label{sec:org4a04fd6}

:TODO:

\subsection{{\bfseries\sffamily TODO} Table of contents}
\label{sec:org6730055}

\begin{scriptsize}
\begin{itemize}
\item \hyperref[sec:org5f263db]{Preamble}
\end{itemize}
\end{scriptsize}

\section{Introduction}
\label{sec:org44c2a58}

This section introduces the concepts of \emph{types},
a particularly useful language safety feature.

Common simple types and methods of building new types are discussed,
as well as some more advanced topics.

\section{Properties of type systems}
\label{sec:orgb02556e}

In the previous notes, we have discussed
\begin{itemize}
\item polymorphism and
\item static/dynamic typing
\end{itemize}
which are two important properties of a type system.

Here we discuss some other commonly discussed properties,
before discussing in the following sections
what is arguably the most important property:
what types a language might have.

\subsection{“Strong” and “weak” typing}
\label{sec:org03b9cdc}

These are comparative terms.
\begin{itemize}
\item We'll consider them a subjective criteria.
\end{itemize}

“Strongly typed”
\begin{itemize}
\item Languages are frequently called strongly typed.
\begin{itemize}
\item But less frequently do they state what they mean by that.
\item The term is used inconsistently.
\begin{itemize}
\item “C is a strongly typed, weakly checked language”
– Dennis Ritchie, creator of C
\end{itemize}
\end{itemize}
\item We will take it to mean “type clashes are restricted”.
\item That definition does not make a good objective property.
\begin{itemize}
\item What does restricted mean?
\begin{itemize}
\item Is it a warning or an error?
\item Does type casting violate this?
\end{itemize}
\item What qualifies as a type clash?
\begin{itemize}
\item Is implicit type casting allowed?
\end{itemize}
\end{itemize}
\end{itemize}

“Weakly typed” simply means not strongly typed.

\subsection{Explicit and implicit typing}
\label{sec:org3318602}

Languages may require annotations on variables and functions
(\emph{explicit typing}) or allow them to be omitted (\emph{implicit typing}).
\begin{itemize}
\item Implicit typing does not weaken the typing system in any way!
\begin{itemize}
\item A very common misconception.
\end{itemize}
\item In general, type inference is an undecidable problem
(its not guaranteed that the compiler/interpreter can
determine the type).
\begin{itemize}
\item Most languages have relatively simple type systems,
and this is not a problem.
\end{itemize}
\end{itemize}

Some languages make type annotations a part of the name,
or annotate names with sigils to indicate type details.
\begin{itemize}
\item In older versions of Fortran, names beginning with
\texttt{i}, \texttt{j} or \texttt{k} were for integer variables,
and all variables were of floating point.
\item In Perl, names beginning with the sigil
\begin{itemize}
\item \texttt{\$} have scalar type,
\item \texttt{@} have array type,
\item \texttt{\%} have hash type, and
\item \texttt{\&} have subroutine type.
\end{itemize}
\end{itemize}

\section{Atomic types}
\label{sec:org94a4e65}

We begin our discussion of what types languages have
with what are usually the “simplest” types: \emph{atomic} types.
\begin{itemize}
\item Atomic in the sense that they cannot be broken down any further.
\item Sometimes called \emph{primitive} or \emph{basic}.
\end{itemize}

Most languages have at least these atomic types.
\begin{itemize}
\item \textbf{Integers}; \texttt{int}
\begin{itemize}
\item Including possibly signed, unsigned, short, and/or long variants.
\end{itemize}
\item \textbf{Floating point} numbers
\begin{itemize}
\item Including possibly single precision and double precision variants.
\end{itemize}
\item \textbf{Characters}
\begin{itemize}
\item Sometimes an alternate name for the byte type (8-bit integers).
\end{itemize}
\item \textbf{Booleans}
\item \textbf{Unit} (the \emph{singleton} type)
\begin{itemize}
\item Sometimes called \texttt{void}, \texttt{nil}-type, \texttt{null}-type or \texttt{none}-type.
\begin{itemize}
\item In C like languages, you cannot store something of type \texttt{void}.
\item Commonly represented as the type of 0-ary tuples,
whose only element is \texttt{()}.
\end{itemize}
\end{itemize}
\item \textbf{Empty}
\begin{itemize}
\item Unlike a singleton type, which has a single value
(called \texttt{nil}, \texttt{null} or \texttt{none}), there is \textbf{nothing} in the empty type.
\end{itemize}
\end{itemize}

\subsection{Implementation of atomic types}
\label{sec:org0e87217}

When we discussed the pure untyped λ-calculus,
we discussed the process of \emph{encoding} the integers and booleans
as functions, since they were not included in the language.
\begin{itemize}
\item We also mentioned that we can add constants for them
to the language, forming an \emph{unpure} untyped λ-calculus.
\begin{itemize}
\item Such added constants are called
\end{itemize}
\end{itemize}

This raises a question we can ask about
“practical” programming languages as well;
\begin{itemize}
\item are the “atomic” (“primitive”, “basic”) types \emph{truly} atomic
(primitive, basic), or are they represented
using one of the language's abstractions?
\item We have discussed the fact that in Scala and Ruby, which we call
“purely object-oriented”, even these atomic types are classes!
\begin{itemize}
\item Whereas in Java and C++, they are not;
there, they are “primitives” which exist
outside the object-oriented abstraction.
\end{itemize}
\end{itemize}

\subsection{Uncommon basic types}
\label{sec:orgfe5a8a9}

Some languages include these less common basic types.
\begin{itemize}
\item \textbf{Complex} numbers
\begin{itemize}
\item Especially for scientific computation.
\end{itemize}
\item \textbf{Decimal} (representation of) numbers
\begin{itemize}
\item Especially for business (monetary) applications.
\item There are decimal numbers that cannot be properly represented
using binary (e.g. \texttt{0.3 = 0.010011}, repeating)
\item Not included in all languages because
they cannot be efficiently represented.
\begin{itemize}
\item It takes at least 4 bits to represent a single decimal digit,
but 4 bits could represent 16 digits, instead of the 10
that are actually possible.
\end{itemize}
\end{itemize}
\end{itemize}

\subsection{Ordinal types}
\label{sec:org63241e0}

Many languages include a means of defining other \emph{finite} types.
Instances include
\begin{itemize}
\item enumeration types (\texttt{enum}'s) and
\item subset/subrange types.
\end{itemize}

\section{Sequences}
\label{sec:org7e8f841}

:TODO:

\section{Algebraic types}
\label{sec:org242f44b}

:TODO:

\section{References}
\label{sec:orge9136ec}

:TODO:

\section{Advanced type systems}
\label{sec:orgff14d16}

:TODO:

\section{Further advanced topics}
\label{sec:org2deb098}

Depending upon time at the end of the course,
we may return to discuss more about types.
\end{document}
