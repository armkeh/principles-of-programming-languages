% Created 2020-09-02 Wed 17:00
% Intended LaTeX compiler: lualatex
\documentclass[11pt]{article}
\usepackage{graphicx}
\usepackage{grffile}
\usepackage{longtable}
\usepackage{wrapfig}
\usepackage{rotating}
\usepackage[normalem]{ulem}
\usepackage{amsmath}
\usepackage{textcomp}
\usepackage{amssymb}
\usepackage{capt-of}
\usepackage{hyperref}
\usepackage{tabularx}
\usepackage{etoolbox}
\makeatletter
\def\dontdofcolorbox{\renewcommand\fcolorbox[4][]{##4}}
\AtBeginEnvironment{minted}{\dontdofcolorbox}
\makeatother
\usepackage[newfloat]{minted}
\usepackage{amsthm}
\theoremstyle{definition}
\newtheorem{definition}{Definition}[section]
\usepackage{unicode-math}
\usepackage{unicode}
\author{Mark Armstrong}
\date{Fall 2020}
\title{Types\\\medskip
\large Principles of Programming Languages}
\hypersetup{
   pdfauthor={Mark Armstrong},
   pdftitle={Types},
   pdfkeywords={},
   pdfsubject={Introduction to types},
   pdfcreator={Emacs 27.0.90 (Org mode 9.3.7)},
   pdflang={English},
   colorlinks,
   linkcolor=blue,
   citecolor=blue,
   urlcolor=blue
   }
\begin{document}

\maketitle

\section{Preamble}
\label{sec:orge41145d}

\subsection{{\bfseries\sffamily TODO} Notable references}
\label{sec:org39e58c2}

:TODO:

\subsection{{\bfseries\sffamily TODO} Table of contents}
\label{sec:org07cf0a1}

\begin{scriptsize}
\begin{itemize}
\item \hyperref[sec:orge41145d]{Preamble}
\end{itemize}
\end{scriptsize}

\section{Introduction}
\label{sec:org0bb2705}

This section introduces the concepts of \emph{types},
a particularly useful language safety feature.

Common simple types and methods of building new types are discussed,
as well as some more advanced topics.

\section{Atomic types}
\label{sec:org967b421}

:TODO:

\section{Sequences}
\label{sec:org43e2a5d}

:TODO:

\section{Algebraic types}
\label{sec:org5f6635b}

:TODO:

\section{References}
\label{sec:orgae353d3}

:TODO:

\section{Advanced type systems}
\label{sec:org241e51a}

:TODO:

\section{Further advanced topics}
\label{sec:org9cbe85d}

Depending upon time at the end of the course,
we may return to discuss more about types.
\end{document}
