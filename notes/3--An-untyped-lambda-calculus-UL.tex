% Created 2020-09-28 Mon 11:26
% Intended LaTeX compiler: lualatex
\documentclass[11pt]{article}
\usepackage{graphicx}
\usepackage{grffile}
\usepackage{longtable}
\usepackage{wrapfig}
\usepackage{rotating}
\usepackage[normalem]{ulem}
\usepackage{amsmath}
\usepackage{textcomp}
\usepackage{amssymb}
\usepackage{capt-of}
\usepackage{hyperref}
\usepackage{tabularx}
\usepackage{etoolbox}
\makeatletter
\def\dontdofcolorbox{\renewcommand\fcolorbox[4][]{##4}}
\AtBeginEnvironment{minted}{\dontdofcolorbox}
\makeatother
\usepackage[newfloat]{minted}
\usepackage{amsthm}
\theoremstyle{definition}
\newtheorem{definition}{Definition}[section]
\usepackage{unicode-math}
\usepackage{unicode}
\author{Mark Armstrong}
\date{Fall 2020}
\title{An untyped λ-calculus, \emph{UL}\\\medskip
\large Principles of Programming Languages}
\hypersetup{
   pdfauthor={Mark Armstrong},
   pdftitle={An untyped λ-calculus, \emph{UL}},
   pdfkeywords={},
   pdfsubject={Our first constructed language; a lambda calculus with no type checking (enforcement).},
   pdfcreator={Emacs 27.0.90 (Org mode 9.4)},
   pdflang={English},
   colorlinks,
   linkcolor=blue,
   citecolor=blue,
   urlcolor=blue
   }
\begin{document}

\maketitle

\section{Preamble}
\label{sec:orgb6efdf0}

\subsection{{\bfseries\sffamily TODO} Notable references}
\label{sec:orgadda15e}

\begin{itemize}
\item Benjamin Pierce,
“\href{https://ebookcentral.proquest.com/lib/mcmu/detail.action?docID=3338823}{Types and Programming Languages}”
\begin{itemize}
\item Chapter 5, The Untyped Lambda-Calculus
\end{itemize}
\end{itemize}

\subsection{{\bfseries\sffamily TODO} Table of contents}
\label{sec:org0a76511}

\begin{scriptsize}
\begin{itemize}
\item \hyperref[sec:orgb6efdf0]{Preamble}
\end{itemize}
\end{scriptsize}

\section{Introduction}
\label{sec:org3714423}

In this section we construct our first simple programming language,
an untyped λ-calculus (lambda calculus).

More specifically, we construct a λ-calculus
without (static) type checking (enforcement),
but including the natural numbers and booleans.

\subsection{What is the λ-calculus?}
\label{sec:org5963cc1}

The λ-calculus is, put simply,
a notation for forming and applying functions.
\begin{itemize}
\item Because the function (procedure, method, subroutine) abstraction
gives us a means of representing control flow,
if we have a means of representing data,
the λ-calculus is a Turing-complete model of computation.
\end{itemize}

\subsection{History}
\label{sec:org9f070d1}

The (basic) λ-calculus as we know it was famously invented
by Alonzo Church in the 1920s.
\begin{itemize}
\item This was one culmination of a great deal of work by
mathematicians investigating the foundations of mathematics.
\end{itemize}

As mentioned, the λ-calculus is a Turing-complete model of computation.
\begin{itemize}
\item Other models proposed around the same time include
\begin{itemize}
\item the Turing machine itself (due to Alan Turing), and
\item the general recursive functions (due to Stephen Cole Kleene.)
\end{itemize}
\item Hence the “Church” in the “Church-Turing thesis”.
\end{itemize}

The λ-calculus has since seen widespread use in the study and design
of programming languages.
\begin{itemize}
\item It is useful both as a simple programming language, and
\item as a mathematical object about which statements can be proved.
\end{itemize}

\subsection{Descendents of the λ-calculus}
\label{sec:orge0fccfd}

:TODO:

\section{The basics}
\label{sec:org5cdcc42}

In our discussion of abstractions, we mentioned
the abstraction of the function/method/procedure/subroutine.
\begin{itemize}
\item The functional abstraction provides a means
to represent control flow.
\end{itemize}

In its pure version, every term in the λ-calculus
is a function.
\begin{itemize}
\item In order for such a system to be at all useful,
it must of course support higher-order functions;
functions may be applied to functions.
\item Values such as booleans and natural numbers
are \emph{encoded} (represented) by functions.
\end{itemize}

\subsection{The terms}
\label{sec:org90085d3}

The pure untyped λ-calculus has just three sort of terms;
\begin{itemize}
\item variables such as \(x\), \(y\), \(z\),
\item \emph{λ-abstractions}, of the form \(λ x ∙ t\),
\begin{itemize}
\item where \(x\) is a variable and \(t\) is a λ-term, and
\end{itemize}
\item applications of the form \(t u\)
\begin{itemize}
\item where \(t\) and \(u\) are λ-terms.
\end{itemize}
\end{itemize}

The meaning of each term is, informally:
\begin{itemize}
\item A λ-abstraction \(λ x ∙ t\) represents a function of one argument,
which, when applied to a term \(u\), substitutes
all occurrences of \(x\) with \(u\).
\item An application applies the term \(u\) to the function (term) \(t\).
\item A variable on its own (a free variable) has no further meaning.
\begin{itemize}
\item Variables are intended to be \emph{bound}.
\end{itemize}
\end{itemize}

\subsection{Variable binding}
\label{sec:org6552258}

:TODO:

\section{The formal syntax and semantics of \emph{UL}}
\label{sec:orgbc3c36b}

\subsection{A grammar for \emph{UL}}
\label{sec:org8f56a83}

\begin{minted}[breaklines=true]{text}
⟨term⟩ ∷= var | λ var • ⟨term⟩ | ⟨term⟩ ⟨term⟩
\end{minted}

In the case that we are restricted to ASCII characters,
we will write abstraction as
\begin{minted}[breaklines=true]{text}
“lambda” var . ⟨term⟩
\end{minted}

\section{α-conversion, β-reduction and η-conversion}
\label{sec:org62c600a}

:TODO:

\section{Topics of theoretical interest}
\label{sec:org5722189}

\subsection{The pure λ-calculus}
\label{sec:org869f696}

:TODO:

\subsection{Nameless representation of terms}
\label{sec:orgd8d36a8}

:TODO:
\end{document}
