% Created 2020-11-27 Fri 13:31
% Intended LaTeX compiler: lualatex
\documentclass[11pt]{article}
\usepackage{graphicx}
\usepackage{grffile}
\usepackage{longtable}
\usepackage{wrapfig}
\usepackage{rotating}
\usepackage[normalem]{ulem}
\usepackage{amsmath}
\usepackage{textcomp}
\usepackage{amssymb}
\usepackage{capt-of}
\usepackage{hyperref}
\usepackage{tabularx}
\usepackage{etoolbox}
\makeatletter
\def\dontdofcolorbox{\renewcommand\fcolorbox[4][]{##4}}
\AtBeginEnvironment{minted}{\dontdofcolorbox}
\makeatother
\usepackage[newfloat]{minted}
\usepackage{unicode-math}
\usepackage{unicode}
\author{Mark Armstrong}
\date{\today}
\title{Introductory concurrency}
\hypersetup{
   pdfauthor={Mark Armstrong},
   pdftitle={Introductory concurrency},
   pdfkeywords={},
   pdfsubject={An introductory look at concurrency support in our course languages, along with some basic discussion of issues arising from concurrency.},
   pdfcreator={Emacs 27.0.91 (Org mode 9.4)},
   pdflang={English},
   colorlinks,
   linkcolor=blue,
   citecolor=blue,
   urlcolor=blue
   }
\begin{document}

\maketitle
\tableofcontents


\section{Introduction}
\label{sec:orge2e3b2a}
Here we briefly explore (some of) the techniques to take advantage
of concurrency and parallelism in the languages we have used
throughout the course.

Our context for this discussion will be the problem
of computing them maximum element of an (unsorted) tree of integers.
This naturally leads to a divide an conquer approach,
which can easily take advantage of concurrency,
while also avoiding the issues that can arise from concurrent execution.
We do give a brief discussion of these issues before
starting our exploration.

We justify the use of this problem to frame our exploration
with a discussion at the end of the notes.

\section{Some brief background}
\label{sec:org27eb38a}
The following should be review from other courses,
most notably from “Concurrent Systems”.
If you are not familiar with these topics,
this will be a sufficient introduction
for our purposes here.

\subsection{Concurrency vs. parallelism}
\label{sec:orgdc8b0a7}
To define concurrency and parallelism, we must refer to
“tasks being executed”.

Two tasks \texttt{A} and \texttt{B} are said to be \emph{concurrently executed} if
the execution may be carried out in any of these orders:
\begin{itemize}
\item Part of \texttt{A} is executed, then the remainder of \texttt{A} and \texttt{B} are
executed concurrently.
\item Part of \texttt{B} is executed, then the remainder of \texttt{B} and \texttt{A} are
executed concurrently.
\item Part of \texttt{A} and part of \texttt{B} are executed \emph{at the same time},
then the remainder of \texttt{A} and \texttt{B} are executed concurrently.
\end{itemize}

Note that tasks can be run concurrently \emph{even if} it is not possible
for the underlying system to execute two tasks at the same time.

On the other hand, two tasks are said to be \emph{executed in parallel} if
their execution takes place (at least partly) at the same time.

Parallelism then requires hardware support
to run two tasks at the same time.
In the past, this implied distributed systems, but on modern computers,
the presence of multiple \emph{cores} of the processor allow for
local parallel computation.

As the notion of concurrency is more general,
we usually talk about concurrent execution,
unless we specifically mean that tasks are executed at
the exact same time.

\subsection{Process vs. thread}
\label{sec:orgcf261a6}
A \emph{process} is an instance of a program being executed on a machine.
The machine, usually through an operating system,
grants each process its own memory space, and
handles the scheduling of the execution of this and all other processes.

A \emph{thread} is a single sequence of instructions to be executed
by the machine. Threads are not granted their own memory space
by the machine (but the process may handle assigning memory to threads.)

A single process process may be made up of several threads;
so long as the process is being executed, there is at least
one thread (of execution.)

\subsection{Race condition vs. deadlock vs. starvation}
\label{sec:orga3c5c05}
Race conditions, deadlock and starvation are all classes of error
that can occur when using concurrency.

A \emph{race condition} refers to an error that arises when …

A \emph{deadlock} refers to an error that arises when …

A task is said to \emph{starve} when …

\section{Creating threads with futures in Clojure}
\label{sec:org50e780e}
Clojure has support for several approaches to concurrency,
including but not limited to:
\begin{itemize}
\item a STM (Software Transactional Memory) system,
\item an \emph{agent} system (somewhat related to the \emph{actor model}), and
\item an \emph{atom} system which avoids race conditions
regarding changes to data.
\end{itemize}
Refer to the \href{https://clojure.org/about/concurrent\_programming}{documentation}
for more information about these.

For today, we are well served with a simpler approach:
the \texttt{future} macro.
A sequence of expressions wrapped in a \texttt{future} are evaluated in
a new thread.
\begin{minted}[breaklines=true]{clojure}
(future (Thread/sleep 4000) ;; 4 seconds
        (println "Computation on this thread paused and this printed after 4 seconds."))

(println "Computation on the main thread continued and this printed immediately.")
\end{minted}

Before we look closer at futures,
we must discuss how we represent these trees in Clojure.
Then we will show a single-threaded implementation of the maximum
function, and finally rework it into an approach using threads via futures.

\subsection{Representing trees in Clojure}
\label{sec:orge2074c0}
Quite simply, we will consider \emph{nested sequences} to be trees.
We will use the convention (at least for these notes)
that the first element of the list is the \emph{label} of a node,
and the next two elements of the list gives the \emph{children} of that node.

For instance, the tree
\includegraphics{../../homework/media/BinTree.png}
can be written
\begin{minted}[breaklines=true]{clojure}
'(1 (2) (3 (4) (5)))
\end{minted}

Note that in the case that a tree has no children,
we omit the rest of the list.

\subsection{Finding the maximum of a tree}
\label{sec:orgfb6cf19}
\begin{minted}[breaklines=true]{clojure}
(defn maximum-tree
  "Find the maximum element of a tree of integers `t`.
`t` is assumed to be a list whose first element is an integer
(the label of the root) and whose remaining elements are lists
representing trees containing integers."
  [t]
  (if
    (empty? t) ##-Inf
    ;; Use list deconstruction to separate the `label` and the list of `children`.
    (let [[label left right] t]
      (max label (maximum-tree left) (maximum-tree right)))))
\end{minted}

\begin{minted}[breaklines=true]{clojure}
(maximum-tree [1 [2] [3 [4] [5]]])
\end{minted}

\subsection{Using futures}
\label{sec:org409f0e4}
:TODO:

\section{Forking processes in Ruby}
\label{sec:org9bbdf09}
:TODO:

\section{Justifying our example problem}
\label{sec:orgf4ac4ab}
:TODO:
\end{document}
