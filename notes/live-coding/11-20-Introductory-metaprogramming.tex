% Created 2020-11-23 Mon 12:31
% Intended LaTeX compiler: lualatex
\documentclass[11pt]{article}
\usepackage{graphicx}
\usepackage{grffile}
\usepackage{longtable}
\usepackage{wrapfig}
\usepackage{rotating}
\usepackage[normalem]{ulem}
\usepackage{amsmath}
\usepackage{textcomp}
\usepackage{amssymb}
\usepackage{capt-of}
\usepackage{hyperref}
\usepackage{tabularx}
\usepackage{etoolbox}
\makeatletter
\def\dontdofcolorbox{\renewcommand\fcolorbox[4][]{##4}}
\AtBeginEnvironment{minted}{\dontdofcolorbox}
\makeatother
\usepackage[newfloat]{minted}
\usepackage{unicode-math}
\usepackage{unicode}
\author{Mark Armstrong}
\date{\today}
\title{Introductory metaprogramming}
\hypersetup{
   pdfauthor={Mark Armstrong},
   pdftitle={Introductory metaprogramming},
   pdfkeywords={},
   pdfsubject={A very introductory look at metaprogramming.},
   pdfcreator={Emacs 27.0.90 (Org mode 9.4)},
   pdflang={English},
   colorlinks,
   linkcolor=blue,
   citecolor=blue,
   urlcolor=blue
   }
\begin{document}

\maketitle
\tableofcontents


\section{Metaprogramming}
\label{sec:org2220db9}
Metaprogramming, as the name implies,
regards programs \emph{that are constructed dynamically},
that is, programs whose construction is (at least partially)
carried out at runtime.

\section{Disclosure}
\label{sec:org1283a49}
I am not an expert on metaprogramming; far from it.

This material is intended as a \emph{very introductory} taste of what
metaprogramming offers.

It is \emph{not} intended to show particularly \emph{good} uses of metaprogramming.
In particular, the Ruby examples might be somewhat dangerous and
ill-advised (the Clojure examples should be a little higher in quality.)

\section{Ruby}
\label{sec:org559f97b}
Suppose we want to have a way to run a piece of code on an object
twice. We can do this for a particular class fairly easily.

As a first attempt:
\begin{minted}[breaklines=true]{ruby}
class MyClass
  def to_s
    "I am an object"
  end

  def twice(&block)
    yield
    yield
  end
end

x = MyClass.new
puts x.twice { puts self }
\end{minted}

In this first attempt, the \texttt{self} in the block refers to \texttt{self} at the
point the block is written.

We can make use of the \texttt{instance\_eval} method instead of \texttt{yield} to
run the block \emph{within the method we are defining}.
\begin{minted}[breaklines=true]{ruby}
class MyClass
  def to_s
    "I am an object"
  end

  def twice(&block)
    instance_eval &block
    instance_eval &block
  end
end

x = MyClass.new
puts x.twice { puts self }
\end{minted}

This is fine for this kind of object. But I can't use it more generally;
for instance, I can't do
\begin{minted}[breaklines=true]{ruby}
5.twice { puts self }
\end{minted}

What I need to do is add this method to the \texttt{Object} class
that all other classes inherit from.
\begin{minted}[breaklines=true]{ruby}
class Object

  def twice(&block)
    instance_eval &block
    instance_eval &block
  end
end

5.twice { puts self }
\end{minted}

We can go further. For instance, we can modify how Ruby responds
if we call a method that does not exist for a given object.
\begin{minted}[breaklines=true]{ruby}
class Object

  def method_missing(method_name, *args, &block)
    print "Uh-oh; I couldn't find the "
    print method_name
    puts " method!"
    return nil
  end
end

5.do_something
puts "Look ma, I didn't crash!"
\end{minted}

\section{Clojure}
\label{sec:orgf7f3395}
In Clojure, we can use \emph{macros} to perform metaprogramming.

For instance, we might miss the \texttt{unless} construct which is
present in many languages, but not in Clojure.
We could try to define it as a function:
\begin{minted}[breaklines=true]{clojure}
(defn unless [test then else]
  (if test
    else
    then))
\end{minted}

\begin{minted}[breaklines=true]{clojure}
(unless (= 0 0)
        (println "Equality is broken!")
        (println "Nevermind, it's okay."))
\end{minted}

We can specify the behaviour of \texttt{unless} more precisely with a macro.
\begin{minted}[breaklines=true]{clojure}
(defmacro unless [test then else]
  `(if ~test
     ~else
     ~then))
\end{minted}
\end{document}
