% Created 2020-12-02 Wed 11:26
% Intended LaTeX compiler: lualatex
\documentclass[11pt]{article}
\usepackage{graphicx}
\usepackage{grffile}
\usepackage{longtable}
\usepackage{wrapfig}
\usepackage{rotating}
\usepackage[normalem]{ulem}
\usepackage{amsmath}
\usepackage{textcomp}
\usepackage{amssymb}
\usepackage{capt-of}
\usepackage{hyperref}
\usepackage{tabularx}
\usepackage{etoolbox}
\makeatletter
\def\dontdofcolorbox{\renewcommand\fcolorbox[4][]{##4}}
\AtBeginEnvironment{minted}{\dontdofcolorbox}
\makeatother
\usepackage[newfloat]{minted}
\usepackage{amsthm}
\theoremstyle{definition}
\newtheorem{definition}{Definition}[section]
\usepackage{unicode-math}
\usepackage{unicode}
\author{Mark Armstrong}
\date{Fall 2020}
\title{Imperativeness and an imperative core \emph{WHILE}\\\medskip
\large Principles of Programming Languages}
\hypersetup{
   pdfauthor={Mark Armstrong},
   pdftitle={Imperativeness and an imperative core \emph{WHILE}},
   pdfkeywords={},
   pdfsubject={Discussion of traits of the imperative programming paradigm not found in pure functional languages. a small imperative language.},
   pdfcreator={Emacs 27.0.91 (Org mode 9.4)},
   pdflang={English},
   colorlinks,
   linkcolor=blue,
   citecolor=blue,
   urlcolor=blue
   }
\begin{document}

\maketitle

\section{Preamble}
\label{sec:orged3510b}
\subsection{{\bfseries\sffamily TODO} Notable references}
\label{sec:orgb8689bc}
:TODO:

\subsection{{\bfseries\sffamily TODO} Table of contents}
\label{sec:orgeddbb05}
\begin{scriptsize}
\begin{itemize}
\item \hyperref[sec:orged3510b]{Preamble}
\end{itemize}
\end{scriptsize}

\section{Introduction}
\label{sec:org6581f99}
In this section we leave behind the “pure” languages
which we have considered so far, and discuss the central features
of imperative languages.

We then construct a simple imperative language,
based on the \texttt{while} loop,
giving several approaches to defining its semantics
(and extend the language slightly.)

\section{The “Von Neumann Architecture”}
\label{sec:org529b2db}
:TODO:

\section{Imperative traits}
\label{sec:org87f99dd}
:TODO:

\section{The \emph{WHILE} language}
\label{sec:org9936eda}
We now construct a simple imperative language,
and give a stack-machine based semantics for it
followed by a “small-step”, reduction based semantics for it.

After this, we will extend the language slightly
and provide other approaches to its semantics.

To begin, the language consists of
\begin{itemize}
\item integer expressions, which may involve (integer-valued) variables,
\item boolean tests, which can only appear in conditional/iteration constructs, and
\item statements, which include assignment, composition, a conditional \texttt{if},
and the iterating \texttt{while} statement.
\end{itemize}

\subsection{Syntax}
\label{sec:orga8c34ba}
\begin{minted}[breaklines=true]{text}
⟨expr⟩ ∷= constant_integer | variable
        | ⟨expr⟩ ('+' | '*' | '-' | '÷') ⟨expr⟩

⟨test⟩ ∷= ⟨expr⟩ ('=' | '<' | '>') ⟨expr⟩
        | ⟨test⟩ ('and' | 'or') ⟨test⟩

⟨stmt⟩ ∷= 'skip'
        | variable '≔' ⟨expr⟩
        | ⟨stmt⟩ ';' ⟨stmt⟩
        | 'if' ⟨test⟩ 'then' ⟨stmt⟩ 'else' ⟨stmt⟩
        | 'while' ⟨test⟩ 'do' ⟨stmt⟩
\end{minted}

\subsection{A stack-machine semantics for \emph{WHILE}}
\label{sec:org6abbb62}
Constants are simply moved from the control stack to the results stack.
\begin{minted}[breaklines=true]{text}
⟨n ∙ c, r, σ⟩ ↝ ⟨c, n ∙ r, σ⟩
\end{minted}

For variables, we instead place the value of the variable
at the current state onto the results stack.
\begin{minted}[breaklines=true]{text}
⟨v ∙ c, r, σ⟩ ↝ ⟨c, σ(v) ∙ r, σ⟩
\end{minted}

\begin{minted}[breaklines=true]{text}
⟨'E₁ op E₂' ∙ c, r, σ⟩ ↝ ⟨E₂ ∙ E₁ ∙ op ∙ c, σ(v) ∙ r, σ⟩

⟨op ∙ c, n₁ ∙ n₂ ∙ r, σ⟩ ↝ ⟨c, n ∙ r, σ⟩   where n = n₁ op n₂
\end{minted}

\subsection{A small-step semantics for \emph{WHILE}}
\label{sec:orgf0c8785}

\section{The \emph{WHILE} language with scoping}
\label{sec:orgc00abd3}
We now extend our \emph{WHILE} language with the necessary syntax
to express \emph{variable scopes}, and then give
scoping rules for programs.

Our approach introduces both \emph{global} and \emph{local} variables.
\begin{itemize}
\item Global variables may be considered to be the input/output to the program
in this model.
\end{itemize}

\subsection{Syntax}
\label{sec:orgc3ff243}

\begin{minted}[breaklines=true]{text}
⟨expr⟩ ∷= constant_integer | variable
        | ⟨expr⟩ ('+' | '*' | '-' | '÷') ⟨expr⟩

⟨test⟩ ∷= ⟨expr⟩ ('=' | '<' | '>') ⟨expr⟩
        | ⟨test⟩ ('and' | 'or') ⟨test⟩

⟨stmt⟩ ∷= 'skip'
        | 'local' variable 'in' ⟨stmt⟩
        | variable '≔' ⟨expr⟩
        | ⟨stmt⟩ ';' ⟨stmt⟩
        | 'if' ⟨test⟩ 'then' ⟨stmt⟩ 'else' ⟨stmt⟩
        | 'while' ⟨test⟩ 'do' ⟨stmt⟩

⟨prog⟩ ∷= ⟨globals⟩ ⟨stmt⟩

⟨globals⟩ ∷= 'global' { variable }
\end{minted}

\subsection{The scoping rules}
\label{sec:orgf92d1f4}

:TODO:

\subsection{A small-step semantics for \emph{WHILE} with scoping}
\label{sec:org53cd557}

:TODO:

\subsection{A \emph{big-step} semantics for \emph{WHILE} with scoping}
\label{sec:org79a7913}

:TODO:
\end{document}
