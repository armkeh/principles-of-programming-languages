% Created 2020-11-16 Mon 12:26
% Intended LaTeX compiler: lualatex
\documentclass[11pt]{article}
\usepackage{graphicx}
\usepackage{grffile}
\usepackage{longtable}
\usepackage{wrapfig}
\usepackage{rotating}
\usepackage[normalem]{ulem}
\usepackage{amsmath}
\usepackage{textcomp}
\usepackage{amssymb}
\usepackage{capt-of}
\usepackage{hyperref}
\usepackage{tabularx}
\usepackage{etoolbox}
\makeatletter
\def\dontdofcolorbox{\renewcommand\fcolorbox[4][]{##4}}
\AtBeginEnvironment{minted}{\dontdofcolorbox}
\makeatother
\usepackage[newfloat]{minted}
\usepackage{unicode-math}
\usepackage{unicode}
\author{Mark Armstrong}
\date{\today}
\title{Tail recursion in Clojure}
\hypersetup{
   pdfauthor={Mark Armstrong},
   pdftitle={Tail recursion in Clojure},
   pdfkeywords={},
   pdfsubject={Visualising stack frames in recursive definitions in Clojure in order to understand how tail recursion saves memory.},
   pdfcreator={Emacs 27.0.91 (Org mode 9.4)},
   pdflang={English},
   colorlinks,
   linkcolor=blue,
   citecolor=blue,
   urlcolor=blue
   }
\begin{document}

\maketitle
\tableofcontents


\section{Introduction}
\label{sec:orgf0f77dc}
These notes were created during the tutorials on November 16th and 18th.

\section{Motivation}
\label{sec:org5a5b318}
In general, recursion as a control structure is \emph{less efficient} than
iterating (or looping) constructs, because each recursive call
requires an additional \emph{stack frame}, used to store the parameters
and any local memory for the call.

This is why “uncontrolled recursion” usually results in a \emph{stack overflow},
that is, running out of memory on the stack,
whereas an “infinite loop” may continue to run indefinitely,
never causing memory issues.

However, while these additional stack frames
are necessary \emph{in general} for recursion,
there is a special case in which the stack frame can be \emph{reused},
avoiding this issue. This kind of recursion is known
as \emph{tail recursion}, and our goal today is to visualise how it works,
as well as come to understand how to write tail recursive functions.

We will use Clojure for this exercise,
both because it is a functional language and hence recursion
is a very natural control structure to use in it,
and because its powerful \emph{macro} system will allow us
to provide you with a new way to define functions
which “automatically” output a visualisation of
their use of the stack.

\section{Recursion and the stack}
\label{sec:org0e2e930}
\section{Tail recursion}
\label{sec:org6f45153}
\section{Visualising the stack}
\label{sec:org5da1a41}
\section{A factorial function that shows its stack usage}
\label{sec:org0561024}
\section{A tail-recursive factorial function that shows its stack usage}
\label{sec:orga73311c}
\section{A macro to automate showing stack usage}
\label{sec:org211c522}
\end{document}
