% Created 2020-11-25 Wed 10:28
% Intended LaTeX compiler: lualatex
\documentclass[11pt]{article}
\usepackage{graphicx}
\usepackage{grffile}
\usepackage{longtable}
\usepackage{wrapfig}
\usepackage{rotating}
\usepackage[normalem]{ulem}
\usepackage{amsmath}
\usepackage{textcomp}
\usepackage{amssymb}
\usepackage{capt-of}
\usepackage{hyperref}
\usepackage{tabularx}
\usepackage{etoolbox}
\makeatletter
\def\dontdofcolorbox{\renewcommand\fcolorbox[4][]{##4}}
\AtBeginEnvironment{minted}{\dontdofcolorbox}
\makeatother
\usepackage[newfloat]{minted}
\usepackage{unicode-math}
\usepackage{unicode}
\author{Mark Armstrong}
\date{\today}
\title{Introduction to Clojure}
\hypersetup{
   pdfauthor={Mark Armstrong},
   pdftitle={Introduction to Clojure},
   pdfkeywords={},
   pdfsubject={An introduction to Clojure, a language in which programs themselves are written as data within the language.},
   pdfcreator={Emacs 27.0.91 (Org mode 9.4)},
   pdflang={English},
   colorlinks,
   linkcolor=blue,
   citecolor=blue,
   urlcolor=blue
   }
\begin{document}

\maketitle
\tableofcontents


\section{Introduction}
\label{sec:org0447adb}
These notes were created for, and in some parts \textbf{during},
the lecture on November 6th and the following tutorials.

\section{Motivation}
\label{sec:orge443038}
:TODO:

\section{Getting Clojure}
\label{sec:orga0bc1c3}
For a quick start with Clojure, you can use
\href{https://repl.it/languages/clojure}{repl.it}.

For instructions on installing Clojure,
see the Clojure \href{https://clojure.org/guides/getting\_started}{getting started guide}.

For the purposes of testing in Docker,
we will use the build tool \href{https://leiningen.org/}{Leiningen}.

That said, we do not assume that a “project” is created
for the homeworks and assignments. As with Scala,
we will assume that your Clojure code is a \emph{script},
to be run in the REPL.

Specifically, we will run your code by starting a REPL with \texttt{lein repl},
and then “dumping” your code into that REPL.

If you want to check how your code runs
with \texttt{lein repl} on your own system,
you can run \texttt{cat my-code-file.clj | lein repl}.
This is the command the Docker image will use.

\section{The syntax and (most of) the semantics of Clojure}
\label{sec:org03d183c}
The syntax of Lisps such as Clojure tend to be extremely minimal.
For today at least, we will work with a subset of the language
described by the following grammar, which is sufficient
for a fair amount of programming.
\begin{minted}[breaklines=true]{text}
⟨expr⟩ ::= number
         | "nil"
         | ⟨list⟩
         | ⟨array⟩
         | symbol

⟨list⟩ ∷= "(" {⟨expr⟩} ")"

⟨array⟩ ∷= "[" {⟨expr⟩} "]"
\end{minted}

For example, the following are all Clojure expressions.
\begin{itemize}
\item Integers.
\begin{minted}[breaklines=true]{clojure}
2
-1
\end{minted}
\item Symbols
\begin{minted}[breaklines=true]{clojure}
:symbols
:a
:b
\end{minted}

\item Lists (note the \texttt{'} or \texttt{quote} usage; we will explain it below in \ref{sec:org7a3684e}.)
\begin{itemize}
\item Which are heterogeneous.
\end{itemize}
\begin{minted}[breaklines=true]{clojure}
;; Lists
()
'(1 2 3)
'(:a 2 3)
(quote (1 2 3))
'((1 2) (3 4))

(first '(1 2 3)) ;; => 1
                 ;; first is often called car in other Lisps.
                 ;; Standing for "Contents of Address part of Register",
                 ;; referring to the memory layout used in the original implementation of Lisp.
(rest '(1 2 3)) ;; => (2 3)
                ;; rest is often called cdr in other Lisps.
                ;; (Pronounced "could-er".)
                ;; Standing for "Contents of Decrement part of Register".
\end{minted}

\item Arrays
\begin{itemize}
\item Also heterogeneous.
\end{itemize}
\begin{minted}[breaklines=true]{clojure}
;; Arrays
[1 2 3]
[:a 1 2]
[]
\end{minted}

\item Sets and maps.
\begin{itemize}
\item Which are also heterogeneous.
\end{itemize}
\begin{minted}[breaklines=true]{clojure}
#{1 2 3}

{:key 1, "my key" :a_value}
\end{minted}
\end{itemize}

Clojure programs are written as lists,
with the head of the list being the \emph{operator} and
the tail of the list being the \emph{operands}.
The (regular) semantics of Clojure expressions
can be described in just two lines;
to evaluate a list,
\begin{enumerate}
\item evaluate each element of the list, and then
\item apply the operands to the operator.
\end{enumerate}
By default, Clojure does use call-by-value semantics.

For instance,
\begin{minted}[breaklines=true]{clojure}
; (1 + 2) * max(3,4)
(* (+ 1 2)    ; (+ 1 2) evaluates to 3
   (max 4 3)  ; (max 4 3) evaluates to 4
   )          ; (* 3 4) evaluates to 12

(+ 1 2 3 4
   5 6 7 8)
(+ 1)
\end{minted}

\section{Special forms; the \texttt{defn}, \texttt{def} and \texttt{fn} forms}
\label{sec:org9f5b001}
When or if the evaluation rules of Clojure given above
prove too limiting, Clojure allows for “special forms”
(pieces of syntax handled differently by the compiler)
to implement constructs.

\subsection{\texttt{defn}}
\label{sec:orgaf65590}
The first of these we will consider is the \texttt{defn} form,
which can be read “define function”.

For instance, here is code which defines two methods,
called \texttt{square} and \texttt{sum\_of\_squares},
and then calls \texttt{sum\_of\_squares} with arguments \texttt{2} and \texttt{3}.
\begin{minted}[breaklines=true]{clojure}
(defn square [x] (* x x))

(defn sum-of-squares [x y]
  (+ (square x)
     (square y)))

(sum-of-squares 2 3)
\end{minted}

Functions can be defined as having different arities.
All the definitions do have to be packaged together as below.
\begin{minted}[breaklines=true]{clojure}
; Define them by "brute force" 
(defn sum-of-squares
  ([x y]   (+ (square x) (square y)))
  ([x y z] (+ (square x) (square y) (square z))))

(sum-of-squares 2 3)
(sum-of-squares 2 3 4)
\end{minted}

They can even accept any number of arguments
by the use of an ampersand \texttt{\&}; an \texttt{\&} followed by a name
indicates the function takes any number of additional arguments,
which are gathered into a list given that name.
\begin{minted}[breaklines=true]{clojure}
;; sum-of-squares-variadic takes any number of arguments,
;; collected into the list xs.
;; (If we wanted to force there to be some arguments,
;; we could put those arguments before the &.)
(defn sum-of-squares-variadic [& xs]
 (if (empty? xs)
   ;; If xs is empty, the sum is 0.
   0
   ;; Otherwise, square the first element of xs, and add it
   ;; to the result of applying sum-of-squares-variadic
   ;; to the rest of the list.
   (+ (square (first xs))
      ;; Note that we need to use apply here as (rest xs) is a list,
      ;; not separate arguments.
      ;; (apply f (x1 x2 … xn)) is equivalent to (f x1 x2 … xn)
      ;; (It also works for arrays of arguments, not just lists.)
      (apply sum-of-squares-variadic (rest xs)))))

(sum-of-squares-variadic 1 2)
(sum-of-squares-variadic 1 2 3 4 5)
(sum-of-squares-variadic)
\end{minted}

\subsection{\texttt{defn} is \texttt{def} plus \texttt{fn}}
\label{sec:org378f244}
The \texttt{defn} form can be thought of
as the combination of the \texttt{def} and \texttt{fn} forms.
The \texttt{def} form defines named values.
\begin{minted}[breaklines=true]{clojure}
(def my-favourite-number 16)
\end{minted}

The \texttt{fn} form defines \emph{anonymous} functions.
We apply this one right away to see its result.
\begin{minted}[breaklines=true]{clojure}
((fn [x] (+ x 1)) my-favourite-number)
\end{minted}

There is a shorthand for the \texttt{fn} syntax; \texttt{(fn [args] body)} can be
replaced by \texttt{\#(body)}, with a small change to \texttt{body}.
There is no specification of argument names with the \texttt{\#()} form,
so they need to be replaced in \texttt{body} with \emph{positional argument names}.
A \texttt{\%} is used if the function has a single parameter, \texttt{\%1}, \texttt{\%2}, etc.
if the function has multiple parameters, and \texttt{\%\&} if it is variadic.
So the above can be written
\begin{minted}[breaklines=true]{clojure}
(#(+ % 1) my-favourite-number)
\end{minted}
(Note; nesting of the \texttt{\#()} form is not allowed, because the
positional parameter names would become ambiguous.)

Of course, given \texttt{def} and \texttt{fn}, we don't need \texttt{defn};
we could just write
\begin{minted}[breaklines=true]{clojure}
(def add-one (fn [x] (+ x 1)))
\end{minted}

We've then given the anonymous function a name we can use later.
\begin{minted}[breaklines=true]{clojure}
(add-one my-favourite-number)
\end{minted}

But \texttt{defn} is more convenient.

\subsection{Scope and the \texttt{let} form}
\label{sec:org4335b42}
The \texttt{def} form creates a new \emph{global} binding.
This means even if not used at the “top-level”,
the binding can be seen anywhere.
\begin{minted}[breaklines=true]{clojure}
(defn get-the-number []
  (def the-number 3)
  the-number)

(get-the-number)

the-number ;; Still in scope
\end{minted}

The same applies for \texttt{defn}.
\begin{minted}[breaklines=true]{clojure}
(defn do-the-thing []
  (defn the-thing [] (+ 1 1))
  (the-thing))

; (the-thing) ;; After definition, the-thing is not yet defined.

(do-the-thing)

(the-thing) ;; But after execution, it's been bound.
\end{minted}

In contrast, the \texttt{let} form creates a new \emph{local} binding.
\begin{minted}[breaklines=true]{clojure}
(defn get-secret-symbol []
  (let [get-symbol (fn [] :secret)]
    (get-symbol)))
\end{minted}

\section{The \texttt{quote}, \texttt{'}}
\label{sec:org7a3684e}
Another special form is \texttt{quote}, which is used
when you want to interpret a list as \emph{data} instead
of as a function invokation.
\begin{minted}[breaklines=true]{clojure}
; call function +
(+ 1 2 3)

; create a list
(quote (+ 1 2 3))

; syntactic sugar; just prepend a ' to the front of the list
'(+ 1 2 3)
\end{minted}

\section{Conditional forms}
\label{sec:org4152811}
The form \texttt{if} acts as you would expect; it takes 3 arguments,
the first of which is checked for truth,
and if it is true, the second argument is evaluated.
Otherwise the third argument (if present) is evaluated.
\begin{minted}[breaklines=true]{clojure}
(if (> 7 6) "Yes!" "No!")
(if (< 7 6) "Yes!" "No!")

; The else is optional.
(if (= 7 6) "Not okay!")
(if (not= 7 6) "Okay!")
\end{minted}

All values in Clojure are “true” except for \texttt{nil} and \texttt{false}.
\begin{minted}[breaklines=true]{clojure}
(if 0 "1")
(if () "2")
(if nil "3")
(if false "4")
(if true "5")
\end{minted}

If you only want a “then” branch for your “if”,
then it's best to use the \texttt{when} form.
\begin{minted}[breaklines=true]{clojure}
(when true "hello")
\end{minted}

:TODO: explain
\begin{minted}[breaklines=true]{clojure}
(when true (println "Hello!") (println "Goodbye!"))
\end{minted}

For a more switch-statement like form, use \texttt{cond}.
Since any value (that is not \texttt{false} or \texttt{nil}) is truthy,
we can include an “else” branch by putting a value
as the condition. By convention, the symbol \texttt{:else} is used.
\begin{minted}[breaklines=true]{clojure}
(cond (> 2 5) "2 is greater"
      (< 2 5) "5 is greater"
      :else "They're the same.")
\end{minted}

\section{\texttt{do}, for sequential computation}
\label{sec:org9c29280}
If we put multiple expressions in sequence,
the “result” will usually be the value of the last expression.
For instance, in a function definition:
\begin{minted}[breaklines=true]{clojure}
(defn add [x]
  (println "hello")
  (+ x x))

(add 5)
\end{minted}

In some cases, we cannot put a sequence of expressions
where we might want one. For instance, when using an \texttt{if},
the \texttt{then} branch has to be a single expression.
To get around this, there is a \texttt{do} form for sequencing.
\begin{minted}[breaklines=true]{clojure}
(if true
  (do (println "First")
      (println "Second")))
\end{minted}

There are more complex forms for repeating the same expressions
several times;
\begin{itemize}
\item \texttt{dotimes} for evaluating a certain number of times,
\begin{minted}[breaklines=true]{clojure}
(dotimes [i 3] ;; The name is an iterator
  (println i))
\end{minted}
and

\item \texttt{doseq} for evaluating once for each element in a sequence.
\begin{minted}[breaklines=true]{clojure}
(doseq [i '(1 2 3)] (println i))
\end{minted}
or iterating over multiple sequences
\begin{minted}[breaklines=true]{clojure}
(doseq [i '(1 2 3) j '(:a :b :c)] (println i j))
\end{minted}
\end{itemize}

\section{Side notes}
\label{sec:org615f985}
\subsection{Partial application}
\label{sec:org1b1fc37}
Partial application would be implemented as
a function returning a function,
and invokation of such a function would look like
\begin{minted}[breaklines=true]{clojure}
((f 3) 2)
\end{minted}
\end{document}
