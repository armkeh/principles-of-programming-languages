% Created 2020-09-23 Wed 01:39
% Intended LaTeX compiler: lualatex
\documentclass[11pt]{article}
\usepackage{graphicx}
\usepackage{grffile}
\usepackage{longtable}
\usepackage{wrapfig}
\usepackage{rotating}
\usepackage[normalem]{ulem}
\usepackage{amsmath}
\usepackage{textcomp}
\usepackage{amssymb}
\usepackage{capt-of}
\usepackage{hyperref}
\usepackage{tabularx}
\usepackage{etoolbox}
\makeatletter
\def\dontdofcolorbox{\renewcommand\fcolorbox[4][]{##4}}
\AtBeginEnvironment{minted}{\dontdofcolorbox}
\makeatother
\usepackage[newfloat]{minted}
\usepackage{unicode-math}
\usepackage{unicode}
\author{Mark Armstrong}
\date{\today}
\title{Introduction to Prolog}
\hypersetup{
   pdfauthor={Mark Armstrong},
   pdftitle={Introduction to Prolog},
   pdfkeywords={},
   pdfsubject={An introduction to Prolog and the logical programming paradigm.},
   pdfcreator={Emacs 27.0.90 (Org mode 9.3.8)},
   pdflang={English},
   colorlinks,
   linkcolor=blue,
   citecolor=blue,
   urlcolor=blue
   }
\begin{document}

\maketitle
\tableofcontents


\section{Introduction}
\label{sec:org39e7144}
These notes were created for, and in some parts \textbf{during},
the lecture on September 18th and the following tutorials.

\section{Motivation}
\label{sec:org0cd1665}
Today, we begin investigating another non-imperative paradigm
other than functional programming: \emph{logical programming}.

As we've seen in the previous hands-on lecture,
functional programming takes advantage of \emph{immutability} in order
to make reasoning about programs easier.
It also focuses on \emph{compositionality}, including
in its use of \emph{higher-order functions}, all of which
makes programming very much like writing functions in mathematics
(where there is no mutable state).

Logical programming also assumes immutability,
but instead of compositionality as a method of computation,
it uses a (Turing-complete) subset of first-order predicate logic.
Programs are databases of \emph{inference rules} describing the problem domain,
and programs are initiated by \emph{queries} about the problem domain which
the system tries to prove are true (a logical consequence of the rules)
or false (refutable by the rules).

To put it simply, in logical programming, you describe the problem,
rather than the solution.

\section{A note about fonts}
\label{sec:org9f777a2}
In these notes, I use plaintext blocks in order to write the
inference rules, and use em-dashes (—) to create the horizontal rules.

In some cases, the em-dashes may not show quite correctly.
For instance, in the PDF version of these notes, there is
a small space between each dash. In some browsers, they
may not show at all (they work in my install of Chrome, at least.)

Apologies for any issues reading these notes caused by this.

\section{(Re)introduction to inference rules}
\label{sec:org45143a1}
Recall: an inference rule has the form
\begin{minted}[breaklines=true]{text}
 Premise₁   Premise₂   …   Premiseₙ
————————————————————————————————————— Rule Name
            Conclusion
\end{minted}
where \texttt{Premise₁}, \texttt{Premise₂}, …, and \texttt{Premiseₙ} are some statements in our
domain, and \texttt{Conclusion} is a statement
that can be concluded from the premise statements.

In the domains of logics, the statements range over formulae
(i.e., boolean expressions built up from
boolean constants, predicates and propositional connectors),
and we may have rules such as
\begin{minted}[breaklines=true]{text}
  P     Q
———————————— ∧-Introduction
   P ∧ Q
\end{minted}
which says “given \texttt{P} and \texttt{Q}, we may conclude \texttt{P ∧ Q}”,
\begin{minted}[breaklines=true]{text}
   P ∧ Q
——————————— ∧-Eliminationˡ
     P
\end{minted}
which says “given \texttt{P ∧ Q}, we may conclude \texttt{P}”, and
\begin{minted}[breaklines=true]{text}
  P     P ⇒ Q
——————————————— Modus Ponens
     Q
\end{minted}
which says, by translating the \texttt{⇒} to English,
“given \texttt{P} and if \texttt{Q} follows from \texttt{P}, then we can conclude \texttt{Q}”.

(Technically, these are \emph{rule schemas};
the \emph{meta-variables} \texttt{P} and \texttt{Q} can be instantiated
to obtain specific rules.)

Note that in these rules, we have the following \emph{meta-syntax}:
\begin{enumerate}
\item Whitespace between premises is understood as a form of conjunction.
\item The horizontal rule is understood as a form of implication.
\end{enumerate}

Any rule which does not have a premise is called an \emph{axiom};
axioms are the known results of our domain, which do not need to be proven.
For instance,
\begin{minted}[breaklines=true]{text}
————————— ⊤-Introduction
  true
\end{minted}

A collection of inference rules (or rule schemas)
and axioms gives us a \emph{proof system}.

See, for instance,
the \href{https://cs.uwaterloo.ca/\~david/cl/natural-deduction-pfenning.pdf}{natural deduction} proof calculus
for classical logic.

You have likely seen the \emph{equational} approach to proofs
which is favoured by Gries and Schneider's
“A Logical Approach to Discrete Math”, and used in
\href{https://www.cas.mcmaster.ca/\~kahl/CS2DM3/2020/}{CS/SE 2dm3 at McMaster}
using the \href{http://calccheck.mcmaster.ca/}{CalcCheck tool}.
Proof systems are an alternate approach to proof;
see \href{https://alhassy.github.io/CalcCheck/LectureNotes.html\#Rules-of-Equality-and-Proof-Trees-vs-Calculational-Proofs}{Musa's notes on the relationship} from this year's 2dm3.

Inside of a proof system, we may construct proofs of statements
via \emph{proof trees}, which are trees where every node is a statement,
and the connections between nodes correspond to the use of rules.

For instance, we have the following silly proof of \texttt{true} which
uses the rules given above.
\begin{minted}[breaklines=true]{text}
—————— ⊤-Introduction        —————— ⊤-Introduction
 true                         true
——————————————————————————————————— ∧-Introductionˡ
          true ∧ true
       ————————————————— ∧-Eliminationˡ
             true
\end{minted}

Notice, by convention, we write proof trees
from the \textbf{bottom up}.
The root, at the bottom, is what we intend to prove.
The leaves, at the top, must either
\begin{enumerate}
\item be axioms, or
\item be local assumptions.
\end{enumerate}

Proof trees may be read from the top down,
to see how the conclusion follows from the axioms and assumptions.
It is generally better to read from the bottom up, though;
otherwise the proof often seems to make
unwarranted steps, or informally, it “pulls a rabbit from its hat”.

\section{Inference rules in other domains}
\label{sec:orga95765d}
The use of inference rules is not limited to the domain of logics.

It is perhaps better not to think of inference rules
as defining a \emph{proof system} (which makes us think
of truth values and logics),
but as defining a \emph{game}: starting from
the rules and axioms, what can we obtain?

For instance, consider the following problem.

\subsection{The two bucket problem}
\label{sec:org2e4c8d8}
\subsubsection{The problem}
\label{sec:org49f6d66}
Suppose you are given two buckets,
\begin{itemize}
\item one of which holds 5 units of water, and
\item one of which holds 3 units of water.
\end{itemize}

You are tasked with collecting exactly 4 units of water;
no more, and no less.
You begin with 0 units in both buckets.

You may at any point
\begin{itemize}
\item fill one bucket entirely from a tap,
\item pour the water out of a bucket, emptying it entirely, or
\item pour one bucket into another until either the first is empty
or the second is full.
\end{itemize}

You are tasked with collecting exactly 4 units of water
using only those three kinds of actions.

\subsubsection{The rules}
\label{sec:org86deef5}
Let us represent the state of the bucket by a pair of numbers.
In these rules, we will use
\begin{itemize}
\item \texttt{x} as a meta-variable for the amount
in the bucket which can hold 5 units, and
\item \texttt{y} as a meta-variable for the amount
in the bucket which can hold 3 units.
\end{itemize}

We can begin only if we have 0 units in both buckets.
\begin{minted}[breaklines=true]{text}

          ––––––––— Start
            0 , 0
\end{minted}

The action of filling a bucket replaces its current amount
with the maximum amount.
\begin{minted}[breaklines=true]{text}
  X , Y                  X , Y
––––––––— Fillˡ        ––––––––– Fillʳ
  5 , Y                  X , 3
\end{minted}

The action of emptying a bucket replaces its current amount
with 0.
\begin{minted}[breaklines=true]{text}
  X , Y                  X , Y
––––––––— Emptyˡ       ––––––––– Emptyʳ
  0 , Y                  X , 0  
\end{minted}

:TODO:
\begin{minted}[breaklines=true]{text}
     X , Y        
–—–––––––————–– Pourˡ (provided X + D ≤ 5 ∧ Y - D ≥ 0 ∧ (X + D = 5 ∨ Y - D = 0))
 X + D , Y - D

     X , Y        
––––––––––––––— Pourʳ (provided X - D ≥ 5 ∧ Y + D ≤ 0 ∧ (X - D = 0 ∨ Y + D = 3))
 X - D , Y + D
\end{minted}

\section{Prolog}
\label{sec:orgcad28d4}
\subsection{Programs are databases of inference rules}
\label{sec:orgc1cad92}
Prolog programs are simply databases of inference rules,
also called \emph{clauses}.
An inference rule
\begin{minted}[breaklines=true]{text}
 A₁   A₂   …   Aₙ
––––––––––––––––––––
      B
\end{minted}
is written in Prolog as the clause
\begin{minted}[breaklines=true]{prolog}
b :- a1, a2, …, an.
\end{minted}
(the … is pseudocode, not Prolog syntax)
(notice the period ending the rule).
As with our inference rule, this rule states that
\texttt{b} is true if all of the \texttt{aᵢ} are true.
So we can think of \texttt{:-} as \texttt{⇐}, and \texttt{,} as \texttt{∧}.

An axiom
\begin{minted}[breaklines=true]{text}
––––––––––––––
      C
\end{minted}
can be written in Prolog as
\begin{minted}[breaklines=true]{prolog}
c :- true.
\end{minted}
or, more simply,
\begin{minted}[breaklines=true]{prolog}
c.
\end{minted}

\subsection{Get “output” by making queries}
\label{sec:orgafaf846}
To interact with Prolog programs, we make \emph{queries}, to which Prolog
responds by checking its inference rule database to determine
possible answers.

For instance, list catenation in SWI Prolog is a ternary predicate
\begin{verbatim}
append(X,Y,Z)
\end{verbatim}
the rules of which enforce that \texttt{Z} is the result
of catenating \texttt{X} and \texttt{Y}.

So we can make queries about catenation, such as
\begin{verbatim}
% Is [1,2,3,4,5,6] the result of catenating [1,2,3] and [4,5,6]?
?- append([1,2,3], [4,5,6], [1,2,3,4,5,6])
% Response:
% true.


% What are all the possible values of Z for which
% Z is the catenation of [1,2,3] and [4,5,6]?
?- append([1,2,3], [4,5,6], Z)
% Response:
% Z = [1, 2, 3, 4, 5, 6]


% What are the possible values of X and Y so that,
% when they are catenated, the result is [1,2,3,4,5,6]?
?- append(X, Y, [1,2,3,4,5,6])
% Response:
% X = [],
% Y = [1,2,3,4,5,6] ;
% X = [1],
% Y = [2,3,4,5,6] ;
% …
\end{verbatim}

Note: in this way, we get several “functions” from one predicate,
depending upon what question(s) we ask!

\subsection{Names, kinds of terms}
\label{sec:org7b251a2}
In Prolog, any name beginning with an upper case letter
denotes a variable.

Names which begin with lower case letters are \emph{atoms},
which are a type of constant value.
Atoms may be used as the name of predicates.

Character strings surrounded by single quotes, \texttt{'},
are also atoms. So we can write
\begin{verbatim}
'is an empty list'([]).
\end{verbatim}

As you would expect, Prolog also has numerical constants,
such as \texttt{1} or \texttt{3.14}.

Aside from variables and constants,
the remaining kind of Prolog term is a \emph{structure},
which has the form
\begin{verbatim}
atom(term1, …, term1)
\end{verbatim}
(again, the … is pseudocode.)

\subsection{Interacting with Prolog}
\label{sec:orgdf12f0c}
As we've said, a Prolog program consists of clauses
(inference rules.)
For instance
\begin{verbatim}
head(X) :- body(X,Y)
\end{verbatim}
which can be interpreted as having meaning
\begin{minted}[breaklines=true]{text}
∀ X ‌∙ head(X) ⇐ (∃ Y • body(X,Y))
\end{minted}
Then during computation, given this clause and the goal \texttt{head(X)},
the Prolog runtime is tasked with finding a substitution for \texttt{Y}
which makes \texttt{body(X,Y)} true.

We provide Prolog with goals through queries,
usually by loading our programs into the interactive query REPL,
either by running
\begin{minted}[breaklines=true]{shell}
$ swipl my-program.pl
\end{minted}
from the command line, or
\begin{verbatim}
?- consult('my-program.pl')
\end{verbatim}
once running SWI Prolog.
We can also \texttt{assert} or \texttt{retract} rules in the query REPL, if needed.
(Also, use \texttt{listing} or \texttt{listing(name)} to see all given clauses
or clauses about the \texttt{name} predicate.)

\subsection{Unification}
\label{sec:orgd5db950}
The computation model of Prolog involves \emph{unification} of terms.
Terms unify if either
\begin{enumerate}
\item they are equal, or
\item they contain variables that can be instantiated in a way
that makes the terms equal.
\end{enumerate}

So in general, unification involves \emph{searching} for possible
variable bindings, by making use of the clauses
and \emph{modus ponens} (\((P ∧ P ⇒ Q) ⇒ Q\)).

\subsubsection{The goal list}
\label{sec:org7ab4224}
Through this process, the single goal presented by a query
will usually turn into a collection of goals;
for instance, if we query \texttt{?- p(5).} and the search uses a clause
\begin{verbatim}
p(X) :- q(X), r(X).
\end{verbatim}
then we now have as goals \texttt{q(5)} and \texttt{r(5)}.

\subsubsection{Backtracking}
\label{sec:org7203683}
Consider the program
\begin{minted}[breaklines=true]{prolog}
a :- b, c.
a :- b.

b.
c :- false.
\end{minted}

As the Prolog runtime tries to prove \texttt{a}, it will use the first clause,
and \texttt{fail} (because in trying to prove \texttt{c}, it reaches \texttt{false},
which it cannot prove.)
At that point, it has to \emph{backtrack}, and try a different
clause to prove \texttt{a}.

In general, the runtime will backtrack several times
during a proof, and it keeps track of which clauses
have been tried.

:TODO: Aside: contradictory clauses
\begin{verbatim}
a :- false.
a.
\end{verbatim}

\subsubsection{SWI Prolog's search strategy}
\label{sec:org3926926}
\begin{enumerate}
\item Attempt to apply clauses in order from top to bottom
(as in the source code.)
\begin{itemize}
\item Only backtrack and try other clauses
after success or failure.
\end{itemize}
\item Perform a depth first search trying to prove each goal.
\begin{itemize}
\item So if the current goals are \texttt{b} and \texttt{c},
try to prove \texttt{b} before considering \texttt{c}.
\end{itemize}
\end{enumerate}

\subsubsection{Examining the search strategy}
\label{sec:org5fa59c9}
In order to interactively see the process Prolog is using
during unification,
use the \texttt{trace.} command in the REPL.
Then each query will result in a log of calls made and
failures encountered.

\subsection{Equality}
\label{sec:orga92b26f}
Prolog does have an equality comparison , written simply \texttt{=} (not \texttt{==}).
\textbf{However}, this equality does no simplification.
So, for instance, if a variable \texttt{X} has been unified to value \texttt{5},
\begin{verbatim}
X = 5
\end{verbatim}
would be \texttt{true}, but
\begin{verbatim}
X = 2 + 3
\end{verbatim}
would be \texttt{false}.

This non-simplifying equality allows us to consider
the \emph{construction} of terms, rather than just their value.
But in case we want to actually carry out arithmetic
or calculate other values,
we can use the \texttt{is} comparison.
\begin{verbatim}
X is 2 + 3
\end{verbatim}
will evaluate to \texttt{true}.

\subsection{Exerting control over the search; the cut operator}
\label{sec:orgd1555d3}
In part because Prolog's searching mechanism can be naive,
the programmer is given a certain amount of control
over the search.

The most important mechanism for controlling the search
that we will see is the \emph{cut}.

A cut is written \texttt{!}, and can be understood as
“no backtracking is allowed to go beyond this point”.

\subsection{Failure}
\label{sec:org125e4e3}
In order to force a search to fail, we can use a strategic cut
along with the \texttt{false} predicate (which cannot be proven),
as in
\begin{minted}[breaklines=true]{prolog}
p :- !, false.
\end{minted}

When this clause is used, the cut is encountered,
preventing any backtracking.
Then, we immediately state the goal \texttt{false},
which cannot be proven.
So, the runtime must give up here, and return \texttt{false}.

Note that we are not really specifying the return value
by writing \texttt{false}; instead, we are exerting our control
over the search to ensure a \texttt{false} result.

\subsection{Trying to write the two bucket problem}
\label{sec:org31d4f83}
In an ideal world, we would be able to almost directly translate
the above inference rules into Prolog, like so.
\begin{verbatim}
buckets(0,0).
buckets(5,Y) :- buckets(_,Y).
buckets(X,3) :- buckets(X,_).
buckets(0,Y) :- buckets(_,Y).
buckets(X,0) :- buckets(X,_).
buckets(DX,DY) :-
  DX is X + D, DY is Y - D,
  buckets(X,Y),
  DX <= 5,
  DY >= 0,
  DX is 5 or DY is 0.
buckets(DX,DY) :-
  DX is X - D, DY is Y + D,
  buckets(X,Y),
  DX >= 0,
  DY <= 3,
  DX is 0 or DY is 3.
\end{verbatim}
(Note the \texttt{\_} is used where we would have a “singleton” variable,
i.e., a variable which appears nowhere else in a clause.
The \texttt{\_} is simply an anonymous variable name, and using it
reassures Prolog that we didn't mean to refer to another variable.)

But using our knowledge of SWI Prolog's search strategy,
we can quickly see a problem with the order of these clauses.
:TODO:

\subsection{Checking for divisors of a number}
\label{sec:orgaf79093}
\begin{minted}[breaklines=true]{prolog}
hasDivisorLessThanOrEqualTo(_,1) :- !, false.
hasDivisorLessThanOrEqualTo(X,Y) :- 0 is X mod Y, !.
hasDivisorLessThanOrEqualTo(X,Y) :- Z is Y - 1, hasDivisorLessThanOrEqualTo(X,Z).
\end{minted}

\section{Resources}
\label{sec:org1afa97c}
\begin{itemize}
\item \href{https://github.com/alhassy/PrologCheatSheet}{Prolog cheat sheet},
by Musa Al-hassy.
\begin{itemize}
\item Includes links to several other resources.
\end{itemize}
\item \href{https://cseweb.ucsd.edu/\~goguen/courses/130w04/prolog.html}{Lecture notes on Prolog}
by Joseph Goguen, UC San Diego.
\end{itemize}
\end{document}
